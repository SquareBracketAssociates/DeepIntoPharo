% $Author: stef $
% $Date: 2008-04-04 17:14:31 +0200 (Fri, 04 Apr 2008) $
% $Revision: 318 $
%=================================================================
\ifx\wholebook\relax\else
% --------------------------------------------
% Lulu:
    \documentclass[a4paper,10pt,twoside]{book}
    \usepackage[
        papersize={6in,9in},
        hmargin={.75in,.75in},
        vmargin={.75in,1in},
        ignoreheadfoot
    ]{geometry}
    \input{../common.tex}
    \pagestyle{headings}
    \setboolean{lulu}{true}
% --------------------------------------------
% A4:
%   \documentclass[a4paper,11pt,twoside]{book}
%   \input{../common.tex}
%   \usepackage{a4wide}
% --------------------------------------------
    \graphicspath{{figures/} {../figures/}}
    \begin{document}
%   \renewcommand{\nnbb}[2]{} % Disable editorial comments
    \sloppy
\fi



\chapter{Magritte: Meta-data at Work}
\label{cha:magritte}

Many applications consist of a large number of input dialogs and reports 
that need to be built, displayed and validated manually. Often these dialogs 
remain static after the development phase and cannot be changed unless a 
new development effort occurs.  For certain kinds of application domains such 
as small-businesses, changing business plans, modifying workflows, etc. it usually boils down to minor modifications to domain objects and behavior, for example new fields have to be  added, configured differently, rearranged or removed. Performing such tasks is tedious. 

Magritte is a meta-data description framework. With Magritte you describe your domain objects 
and among other things you can get Seaside components and their associated validation for free.  
Moreover Magritte is self-described, enabling the automatic generation of meta-editors which can 
be adapted for building end-user customization of application. 

In this chapter we describe Magritte, its design and how to customize it. Now be warned, Magritte
is a framework supporting meta-data description. As any framework, mastering it takes times. 
It is not a scaffolding engine, therefore Magritte may imply some extra complexity and 
you have to understand when you want to pay for such complexity. %webok



\section{Magritte by Example}
In this section we present the key principles. With such a knowledge you can get 80\% of the power of Magritte
without knowing all its possible customizations. The key idea behind Magritte is the following: given one object with a set of values, and a description of this information, we create tools that can treat such an information. For example automatically create Seaside components. Figure~\ref{fig:mabasicPrinciple} shows that a conference, ESUG'12, instance of the class Conference, is described by a description object. A program (database query, generic ui, seaside component builder) will interpret the value of the instance by using its descriptions. 

\begin{figure}[ht!]
\begin{center}
\includegraphics[width=6cm]{basicPrinciple2}
\caption{An object is described by a description in addition to be defined by its class.\label{fig:mabasicPrinciple}}
\end{center}
\end{figure}

\subsection{Loading Magritte 30}

\ifx\wholebook\relax\else
    \end{document}
\fi



Here are the basic description assumptions:

\begin{itemize}
\item An object is described by adding methods named \ct{description*} (naming convention) to the class-side of its class. Such description methods create different description entities. The following \ct{Address} class method creates a string description object that has a label 'Street', a priority and two accessors street and street: to access it. 

\begin{code}{}
Address class>>#descriptionStreet
	^ MAStringDescription new
		autoAccessor: #street;
		label: 'Street';
		priority: 100;
		yourself
\end{code}

Note that there is no need to have a one to one mapping between the instance variables of the class and the associated descriptions. 

\item All descriptions are automatically collected and put into a container description when sending \ct{description} to the object (see Figure~\ref{fig:container}).

\item Descriptions are defined programmatically and can also be queried. They support the collection protocol (\ct{do:}, \ct{select:}...).
\end{itemize}

\begin{figure}[ht!]
\begin{center}
\includegraphics[width=10cm]{describingAddress}
\caption{An object description grouped automatically in a container.\label{fig:container}}
\end{center}
\end{figure}

\paragraph{Obtaining a component.} Once an object is described, you can obtain a Seaside component by sending to the object
the message \mthind{asComponent}{asComponent}. 

\begin{code}{}
anAddress asComponent
\end{code}

It is often useful to decorate the component with cancel/ok like buttons. This can be done sending the message \mthind{addValidatedForm}{addValidatedForm} to the component. Note that you can also change the label of the validating form using \ct{addValidatedForm:} and passing an array of association whose first element is the message to be sent and the second the label to be displayed.

A Magritte form is generally wrapped with a form decoration via \mthind{addValidatedForm}{addValidatedForm}. Magritte forms don't directly work on your domain objects. They work on a memento of the values pulled from your object using the descriptions. When you call \ct{save},  thevalues are validated using the descriptions, and only after passing all validation rules are the values  committed to your domain object by the momentos via the accessors. This is clearly the place to hook 
a database commit for example.

\begin{code}{}
anAddress asComponent addValidatedForm.

anAddress asComponent addValidatedForm: { #save -> 'Save' . #cancel -> 'Cancel'}
\end{code}


A description container is an object implementing collection behavior (\ct{collect:}, \ct{select:}, \ct{do:}, \ct{allSatisfy:},...). Therefore you can send the normal collection messages to extract the parts of the descriptions you want. You can also iterate over the description or concatenate new ones. Have a look at MAContainer class protocol. 

You can also use the message \mthind{asComponentOn:}{asComponentOn:} aModel to build or select part of a description and get a component
on a given model. The following code schematically shows the idea: two descriptions are assembled and a component
based on these two descriptions is build.

\begin{code}{}
(Address descriptionStreet, Address descriptionPlace) asComponentOn: anAddress)
	addValidatedForm 
\end{code}








\paragraph{Example.} Let us go over a simple but complete example. We want to develop an application to manage (again) person, address and phone number as shown in Figure~\ref{fig:address}. 

\begin{figure}
\begin{center}
\includegraphics[width=8cm]{address}
\caption{Our model. \label{fig:address}}
\end{center}
\end{figure}

We define a class \ct{Address} with four instance variables and their corresponding accessors. 

\begin{classdef}{}
Object subclass: #Address
	instanceVariableNames: 'street place plz canton'
	classVariableNames: ''
	poolDictionaries: ''
	category: 'MaAddress'
\end{classdef}

\begin{method}{An instance of Address}
Address class>>>example1
	"self example1"

	|addr | 
	addr := self new.
	addr plz: 1001.
	addr street: 'Sesame'.
	addr place: 'DreamTown'.
	addr canton: 'Bern'.
	^ addr
\end{method}


Then we add the descriptions to the \ct{Address} class as follows: the street name and the place are  described by a string description, the PLZ is a number with a range between 1000 and 9999,  and since the canton is one of the predefined canton list (our address is for Switzerland so far), we describe it as a single option description. 

\begin{code}{Address}
Address class>>descriptionStreet
	^ MAStringDescription new 
			autoAccessor: #street;
			label: 'Street';
			priority: 100; 
			yourself

Address class>>descriptionPlz
	^ MANumberDescription new 
		autoAccessor: #plz;
		label: 'PLZ';
		priority: 200; 
		beRequired;
		min: 1000 ; 
		max: 9999;
		yourself.

Address class>>descriptionPlace
	^ MAStringDescription new
		autoAccessor: #place;
		label: 'Place'; 
		priority: 300; 
		yourself.

Address class>>descriptionCanton
	^ MASingleOptionDescription new
			autoAccessor: #canton;
			label: 'Canton' ; 
			priority: 400;
			options: #( 'Bern' 'Solothurn' 'Aargau' 'Zuerich' 'Schwyz' 'Glarus');
			beSorted;
			beRequired; 
			yourself.
\end{code}


Now we can start manipulating the descriptions. The first example select the string descriptions.


\begin{code}{Manipulating Descriptions}
| adr |
adr := Address example1.
addr description children select: [:each | each isKindOf: MAStringDescription]
\end{code}


The second example shows how to get automatically the description and the object associated value. 

\begin{code}{}
| adr |
adr := Address example1.
adr description do: [ :description |
		Transcript
			show: description label; show: ':'; tab;
			show: (description toString: (adr readUsing: description));
			cr ]
\end{code}
Executing the second code snippet outputs the following in the Transcript

\begin{code}{}
Street:	Sesame
PLZ:	1001
Place:	DreamTown
Canton:	Bern
\end{code}

\paragraph{Seasiding the Address.}
Now we are ready to get a \Seaside component automatically in a similar manner. 

\begin{classdef}{}
WAComponent subclass: #MyContactAddress
	instanceVariableNames: 'addressForm'
	classVariableNames: ''
	poolDictionaries: ''
	category: 'MaAddress'
\end{classdef}

\begin{figure}[ht]
\begin{center}
\includegraphics[width=8cm]{addressUI}
\caption{Address example1 asComponent}
\end{center}
\end{figure}

\begin{method}{Seasiding the Address}
MyContactAddress>>>initialize
	super initialize.
	addressForm := Address example1 asComponent

MyContactAddress>>>children
	^ Array with: addressForm
\end{method}


Now we can decorate the component with validating action working on a memento.
\begin{method}{Seasiding the Address}
MyContactAddress>>>initialize
	super initialize.
	addressForm := Address example1 asComponent addValidatedForm; yourself
\end{method}


\begin{figure}
\begin{center}
\includegraphics[width=8cm]{addressSaveCancel}
\caption{Same magritte generated component with validation.}
\end{center}
\end{figure}




% 
% 
% 
% \begin{code}{Person}
% MAPerson class>>descriptionTitle
% 	^ (MASingleOptionDescription auto: 'title' label: 'Title' priority: 10)
% 		options: #( 'Mr.' 'Mrs.' 'Ms.' 'Miss.' );
% 		yourself.
% 
% MAPerson class>>descriptionFirstName
% 	^ (MAStringDescription auto: 'firstName' label: 'First Name' priority: 20)
% 		beRequired;
% 		yourself.
% 
% MAPerson class>>descriptionLastName
% 	^ (MAStringDescription auto: 'lastName' label: 'Last Name' priority: 30)
% 		beRequired;
% 		yourself.
% \end{code}
% 
% \begin{code}{Relationships}
% MAPerson class>>descriptionHomeAddress
% 	^ (MAToOneRelationDescription auto: 'homeAddress' label: 'Home Address')
% 		classes: (Array with: MAAddressModel);
% 		beRequired;
% 		yourself.
% 
% MAPerson class>>descriptionOfficeAddress
% 	^ (MAToOneRelationDescription auto: 'officeAddress' label: 'Office Address')
% 		classes: (Array with: MAAddressModel);
% 		yourself.
% 
% MAPerson class>>descriptionPicture
% 	^ (MAFileDescription auto: 'picture' label: 'Picture')
% 		addCondition: [ :value | value isImage ] labelled: 'Image expected';
% 		yourself.
% \end{code}


\paragraph{Summary.}
Put your descriptions on the class-side according to the naming-convention.
Ask your object for its description-container by sending \ct{description}.
Get a Seaside component from  your object by sending \ct{asComponent}.



\section{Descriptions}
Descriptions, as we have seen in the above examples, are naturally organized 
in a description hierarchy. A class diagram of the most important descriptions is shown in Figure~\ref{fig:description}. Different kinds of descriptions exist: simple 
type-description that directly map to Smalltalk classes, and some more advanced descriptions that are used to represent a collection of descriptions, or to model relationships between different entities. 


\begin{figure}
	\begin{center}
\includegraphics[width=\linewidth]{descriptionhierarchy}
\caption{Description hierarchy \label{fig:description}}
 \end{center}
\end{figure}

Descriptions are central to Magritte and are connected to accessors and condition that we present below. 
\begin{figure}
	\begin{center}
\includegraphics[width=8cm]{centralGlue}
\caption{Description are a composite and connected via accessors}
\end{center}
\end{figure}

\begin{enumerate}
\item{Type Descriptions.} Most descriptions belong to this group, such as the 
\ct{ColorDescription}, the \ct{DateDescription}, the \ct{NumberDescription}, the 
\ct{StringDescription}, the \ct{BooleanDescription}, etc. All of them describe a 
specific Smalltalk class; in the examples given, this would be Color, Date, 
Number and all its subclasses, String, and Boolean and its two subclasses 
True and False. All descriptions know how to perform basic tasks on those 
types, such as to display, to parse, to serialize, to query, to edit, and to 
validate them. 

\item{Container Descriptions.} If a model object is described, it is often necessary to keep a set of other descriptions within a collection, for example the description of a person consists of a description of the title, the family-name, the birthday, etc. The \ct{ContainerDescription}, and its subclasses, provides 
a collection container for other descriptions. In fact the container implements the whole collection interface, so that users can easily iterate (\ct{do:}), alter (\ct{select:}, \ct{reject:}), transform (\ct{collect:}) and query (\ct{detect:}, \ct{anySatisfy:}, \ct{allSatisfy:}) the containing descriptions. 

\item{Option Descriptions.} The \ct{SingleOptionDescription} describes an entity, for which it is possible to choose up to one item from a list of objects. The \ct{MultipleOptionDescription} describes a collection, for which it is possible to choose any number of items from a predefined list of objects. The 
selected items are described by the referencing description. 

\item{Relationship Descriptions.} Probably the most advanced descriptions 
are the ones that describe relationships between objects. The \ct{ToOneRelationshipDescription} models 
a one-to-one relationship; the \ct{ToManyRelationshipDescription} models a one-to-many relationship using a 
Smalltalk collection. In fact, these two descriptions can also be seen as basic type descriptions, 
since the \ct{ToOneRelationshipDescription} describes 
a generic object reference and the \ct{ToManyRelationshipDescription} describes a collection of object 
references. %The referenced objects are described by the referencing description, 
%which is -- if not manually defined by the developer to -- automatically built from the intersection of the element descriptions. 
\end{enumerate}


\subsection{Exceptions}
Since actions on the meta-model can fail. In addition, cbjects might not match a given meta-model.
Finally it is often important to present to the end users readable error messages. Magritte introduces an exception hierarchy knowing about the description, the failure and a human-readable error message (as shown in Figure\~ref{maexception}).

\begin{figure}[!h]
	\begin{center}
\includegraphics[width=\linewidth]{exceptions}
\caption{The Magritte exception hierarchy. \label{maexception}}
\end{center}
\end{figure}



\section{Adding Validation}

Descriptions can have predicates associated with them to validate the input data. Magritte calls such validation conditions associated to a description. A condition is simple a block returns true or false to validate the data. One is the single field condition which is passed the pending value directly and attached directly to your descriptions. It is as simple as the following block which check the length of the value once the blanks are removed.

\begin{code}{}
addCondition: [:value | value withBlanksTrimmed size > 3] 
\end{code}

There are advantages to having your rules outside your domain objects, especially if you're taking advantage of Magritte as an Adaptive Object Model where users can build their own rules. 
It also allows the mementos to easily test data for validity outside the domain object and gives you a nice place to hook into the validation system in the  components. Still you have to pay attention since it may weaken your domain model. 

\paragraph{Multiple Field Validation.}
When we did the Mini-reservation example in previous chapters we wanted that the starting date is 
later than today. With Magritte we can express this point as follows: 

\begin{code}{}
descriptionStartDate
    ^MADateDescription new 
        selectorAccessor: #startDate;
        label: 'Start Date';
        beRequired;
        !\textbf{addCondition: [:value | value > Date today];}!
        yourself

descriptionEndDate
    ^MADateDescription new 
        selectorAccessor: #endDate;
        label: 'End Date';
        beRequired;
        addCondition: [:value | value > Date today];
        yourself
\end{code}

Now to add a validation that depends on several field values, you have to do a bit more than that because
you cannot from one description access the other. 
Suppose that we want to enforce the endDate should be after the startDate. We cannot add that rule to the 
endDate description because we cannot reference the startDate value. We need to add the rule in a place 
where we can ensure all single field data exists but before it's written to the object from the mementos.

We need to add a rule to the objects container description like this

\begin{code}{}
descriptionContainer 
   ^(super descriptionContainer)
        addCondition: [:memento |
            (memento cache at: self descriptionEndDate) > 
               (memento cache at: self descriptionStartDate)]
        labelled: 'End date must be after start date';
        yourself
\end{code}

This simply intercepts the container after it is built, and adds a multi field validation by accessing the 
potential value of those fields. You get the memento for the object, and all the field values are in 
its cache, keyed by their descriptions. You simply read the values and validate them before they have a 
chance to be written to the real business object. Multi field validations are a bit more complicated than 
single field validations but this pattern works well.



\subsection{Mementos}
Magritte introduces mementos that behave like the original model and that delay modifications until they are proven to be valid. When components read and write to domain objects, they do it using mementos rather than working directly on  the objects. The default memento is \ct{MACheckedMemento}. The mementos give Magritte a place to store invalid form data prior to allowing the data to be committed to the domain object. It also allows Magritte to detect concurrency conflicts. By never committing invalid data to domain objects, there's never a need to roll anything back. 
So mememtos are good since editing might turn a model (temporarily) invalid, canceling an edit shouldn't change the model and concurrent edits of the same model should be detected and (manually) merged.
\mth{mementoClass}{mementoClass} is a class factory that you can specialize to specify your own memento. But normally you should not need that. 



\begin{figure}[!h]
	\begin{center}
\includegraphics[width=10cm]{mementosSimpler}
 \end{center}
\end{figure}


\subsection{Accessors}
In Smalltalk data can be accessed and stored in different ways. Most common data is stored within instance variables and read and written using accessor methods, but sometimes developers choose other strategies, for example to group data within a referenced object, to keep their data stored within a dictionary, or to calculate it dynamically from block closures. Magritte uses a strategy pattern to be able to access the data trough a common interface (see Figure~\ref{fig:accessor}).


\begin{figure}[!h]
\begin{center}
\includegraphics[width=\linewidth]{accessors}
\caption{Accessors in Magritte \label{fig:accessor}}
\end{center}
\end{figure}

By far the most commonly used accessor type is the \ct{SelectorAccessor}. It can be instantiated with two selectors: a zero argument selector to read, and a one argument selector to write. For convenience 
it is possible to specify a read selector only, from which the write selector is 
inferred automatically. 

A special form of the \ct{MASelectorAccessor} is the \ct{MAAutoSelectorAccessor}: it 
automatically creates read accessors, write accessors, and instance variables 
if necessary. This is very useful for fast prototyping, if the  classes 
haven't been fully specified yet; at a later stage of development this accessor 
can be easily replaced with a \ct{MASelectorAccessor}. 

Note that the \ct{descriptionStreet} defined at the beginning of this chapter could be defined using a \ct{selectorAccessor:} message instead of an \ct{autoAccessor:}. 

\begin{code}{}
Address class>>descriptionStreet
	^ MAStringDescription new 
			selectorAccessor: #street;
			label: 'Street';
			priority: 100; 
			yourself
\end{code}

The \ct{MADictionaryAccessor} is used to add and retrieve data from a dictionary 
with a given key. This access strategy is also mainly used for prototyping 
as it allows one to treat dictionaries like objects with object-based instance 
variables. 


% \begin{figure}[!h]
% \begin{center}
% \includegraphics[width=10cm]{readWrite}
% \end{center}
% \end{figure}

When a memento writes valid data to its model, it does so through accessors which are also defined by 
the descriptions. \ct{MAAccessor} subclasses allow you to define how a value gets written to your class. You can redefine the \ct{selectorAccessor:} if you need. 


% Another valuable use for accessors is when you want to flatten several domain objects into a 
% single form. Imagine that  you have a User object with an Address object, and a CreditCard object but you want 
% to edit them all as a single form. You can create your form and dynamically wire up some 
% \ct{MAChainAccessor}s as follows:
% 
% \begin{code}
% userTemplate 
%  ^((((User descriptionEmail, User descriptionPassword, User descriptionConfirmedPassword) 
%   addAll:  (Address description collect: [:each | 
%               each copy accessor: 
%                 (MAChainAccessor accessor: (MASelectorAccessor selector: #address) next: each accessor)]); 
%   addAll: (CreditCard description collect: [:each | 
%    each copy accessor: 
%                (MAChainAccessor accessor: (MASelectorAccessor selector: #creditCard) next: each accessor)])) 
%                 yourself) 
%             asComponentOn: self session user) 
%             addValidatedForm: {(#save -> 'Update Profile')}; 
%             yourself. 
% \end{code}
% The previous code composes the users descriptions with the addresses and credit cards descriptions 
% while chaining the accessors together to allow the addresses fields to be written to the address object 
% rather than directly to the customer object. Note the example of validating form customisation using \ct{addValidatedForm:}. 




\section{Custom Views}
Components control how your objects display. Some  descriptions have an obvious one to one relationship with a UI component while others could easily be shown by several different components. Figure~\ref{custom} shows some components. For example, an MAMemoDescription would be represented by a text area, but a MABooleanDescription could be a checkbox, or a drop down with true and false, or a radio group with true and false. 

Each description defaults to a component, but allows you to override and specify any component you 
choose, including any custom one you may write using the message \mthind{componentClass:}{componentClass:}. 

\begin{code}{setting the class of the description}
aDescription componentClass: aClass
\end{code}

\begin{figure}
\begin{center}
\includegraphics[width=\linewidth]{views}
\caption{Some Magritte specific widgets\label{custom}}
\end{center}
\end{figure}

You can also define your own view by following the steps: 
\begin{itemize}
\item Create a subclass of \ct{MADescriptionComponent}.
\item Override \ct{renderEditorOn:} and/or \ct{renderViewerOn:} as necessary.
\item Use your custom view together with your description by using the accessor \ct{componentClass:}.
\item Possibly add your custom view to its description into \ct{defaultComponentClasses}. 
\end{itemize}


% 
% 
% \section{Customisation in Magritte}
% 
% \subsection{Description Manipulation}
% Instances might want to dynamically filter, add or modify their descriptions.
% Users of a described object often don't need all the available descriptions.
% 
% \paragraph{Solution.} Override \ct{description} on the instance-side to modify the default description-container. Add other methods returning different filtered or modified sets of your descriptions
% 
% \begin{code}{Building descriptions dynamically}
% MAPersonModel>>descriptionPrivateData
% 	^ self description select: [ :each | #( title firstName lastName homeAddress )
% 											includes: each accessor selector ].
% 
% " add another description "
% MAPersonModel>>descriptionWithEmail
% 	^ self description copy
% 		add: (MAStringDescription new) 
% 				autoAccessor: 'email' ;
% 				label: 'E-Mail';
% 				priority: 35; yourself);
% 		yourself.
% 
% " modify existing description "
% MAPersonModel>>descriptionWithRequiredImage
% 	^ self description collect: [ :each |
% 				each accessor selector = #picture
% 					ifTrue: [ each copy beRequired ]
% 					ifFalse: [ each ] ].
% \end{code}
% 
% 
% \begin{code}{Using dynamic description}
% model := MAPersonModel new.
% 
% " get a component"
% component := model descriptionPrivateData
% asComponentOn: model.
% \end{code}
% 
% 
% \subsection{Custom Validation}
% A lot of slightly different validation strategies leads to an explosion of the description class-hierarchy. For example, a number must be in a certain range, an e-mail address must match a regular-expression.
% 
% \paragraph{Solution.}
% Additional validation rules can be added to all descriptions.
% 
% \begin{itemize}
% \item Use \ct{addCondition:labelled:} to add additional conditions to descriptions that will be automatically checked before committing to the model.
% 
% \item The first argument is a block taking one argument, that should return true if the argument validates.
% \item Using a block-closure is possible, but you will loose the possibility to serialize the containing description. Send it the message \ct{asCondition} before adding to parse it and keep it as serialize-able AST within the description. 
% \end{itemize}
% 
% 
% \begin{code}{validation examples}
% ((MANumberDescription new)
% 	autoSelector: #age;
% 	label: 'Age'; yourself)
% 		addCondition: [ :value | value isInteger and: [ value between: 0 and: 100 ] ] 
% 		labelled: 'invalid age';
% 		...
% 
% ((MAStringDescription new)
% 	selectorAccessor: #email;
% 	label: 'E-Mail'; yourself)
% 		addCondition: [ :value | value matches: '#*@#*.#*' ]
% 		labelled: 'invalid e-mail';
% 		...
% 
% ((MADateDescription new)
%  	selectorAccessor: #party;
%  	label: 'Party'; 
% 	yourself)
% 		addCondition: [ :value | self possiblePartyDates includes: value ]
% 		labelled: 'party hard';
% 	...
% \end{code}
% 
% 
\section{Defining your own Descriptions}
In some cases it might happen that there is no description provided to use with a  class.
If your domain manipulates money (amount and currency) or Urls (scheme, domain, port, path, parameters)
you may want to define your own descriptions to take advantage of Magritte. 
 
Extending Magritte is simple, create your own description but remember that Magritte is described within itself so you have to provide certain information.
 
\begin{itemize}
\item Create a subclass of \ct{MAElementDescription}.
\item On the class-side override: \ct{isAbstract} to return false, \ct{label} to return the name of the description. On the instance-side override: \ct{kind} to return the base-class. \ct{acceptMagritte:} to enable visiting and \ct{validateSpecific:} to validate.
\item Create a view, if you want to use it for UI building. 
\end{itemize}
 
\paragraph{Some hints.} We suggest you to have a look at existing descriptions. In addition choosing carefully the right superclass can provide you already part of what you are looking for.  Parsing, printing and (de)serialization is implemented by the following visitors: \ct{MAStringReader}, \ct{MAStringWriter}, \ct{MABinaryReader} and \ct{MABinaryWriter}


 
% \subsection{Custom Views}
% Custom descriptions mostly need a new view. Applications might need a special view for existing descriptions to adapt a better user experience.
% For example Money: an input-field for the amount and a drop-down box to select the currency. 
% 
% \paragraph{Solution.} 
% Choose a different view or create your own.
% 
% \begin{code}{setting the class of the description}
% aDescription componentClass: aClass
% \end{code}
% 
% \begin{figure}
% \includegraphics[width=\linewidth]{views}
% \end{figure}
% 
% You can also define your own view.
% \begin{itemize}
% \item Create a subclass of \ct{MADescriptionComponent}.
% \item Override \ct{renderEditorOn:} and/or \ct{renderViewerOn:} as necessary.
% \item Use your custom view together with your description by using the accessor \ct{componentClass:}.
% \item Possibly add your custom view to its description into \ct{defaultComponentClasses} (there is no clean way to do that right now, Pragmas would help).
% \end{itemize}
% 
% \subsection{Custom Rendering}
% Automatic built UIs are often not that user-friendly, and they all look more or less the same.
% 
% \paragraph{Solution.} Use CSS and customize the rendering of your UI.
% 
% \begin{figure}[!h]
% \includegraphics[width=\linewidth]{customRendering}
% \end{figure}
% 
% \paragraph{Variation 1.}
% 
% \begin{itemize}
% \item Create a subclass of \ct{WAComponent}.
% \item Create an i-var holding onto the automatically built component:      \ct{dialog := aModel asComponent}
% \item Don't forget to return it as a child!
% \item Implement your own rendering code, accessing the magritte sub-views by calling: \ct{dialog childAt: a class descriptionFoo}
% \item Commit your model by sending:   \ct{dialog commit}
% \end{itemize}
% 
% \paragraph{Variation 2.}
% \begin{itemize}
% \item Create a new subclass of \ct{MAComponentRenderer}.
% \item Implement the new visitor to get the layout you need.
% \item Override the method \ct{descriptionContainer} in your model as follows.
% \end{itemize}
% 
% \begin{code}{descriptionContainer}
% My class>>descriptionContainer
% 	^ super descriptionContainer
% 		componentRenderer: MyRendererClass;
% 		yourself.
% \end{code}
% 
% \paragraph{Variation 2.}
% \begin{itemize}
% \item Create a new subclass of MAContainerComponent.
% \item Override the method \ct{renderContentOn:} to get the layout you need (avoiding the vistor).
% \item Override the method \ct{descriptionContainer} in your model like this:
% \end{itemize}
% \begin{code}{descriptionContainer}
% My class>>descriptionContainer
% 	^ super descriptionContainer
% 		componentClass: MyComponentClass;
% 		yourself.
% \end{code}
% 
% 
% 
% 
% 
% 
% 
% 
% 
% \section{Framework Properties}
% 
% Here we present how the framework solves some important problems. Note that you as a end-user
% you do not have to worry about such implementation issues. However knowing how accessors work will help you to understand how to manipulate descriptions. 
% 
% 







% \subsection{Conditions}
% End users want to visually compose conditions. Instances of BlockContext can be hardly serialized.
% Magritte introduces condition objects that can be composed to describe constraints on objects and data.
% 
% \begin{figure}[!h]
% \includegraphics[width=\linewidth]{conditions}
% \end{figure}


\section{Meta Magritte}
Of course Magritte is described in itself, this means that you can use it for building editors
that end users can use. The fun is that you can use the complete infrastructure at its metalevel the same as for your domain objects. Figure~\ref{aare} shows a workflow system developed in Magritte where the end user can edit the forms themselves. 

\begin{figure}[!h]
	\begin{center}
\includegraphics[width=\linewidth]{aare}
\caption{Meta-editors at work\label{aare}}
 \end{center}
\end{figure}


\section{Conclusion}

Now while Magritte is really powerful, using a meta-data system will add another indirection layer to your application, so this is important to understand how to use such a power and when to use plain normal development techniques. A good and small example to learn from is the Conrad conference 
management system developed to manage the ESUG conference. Here Magritte is used to described the different forms and all the data. 


\ifx\wholebook\relax\else
    \end{document}
\fi



%%% Local Variables:
%%% coding: utf-8
%%% mode: latex
%%% TeX-master: t
%%% TeX-PDF-mode: t
%%% ispell-local-dictionary: "english"
%%% End:
