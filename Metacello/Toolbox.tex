% $Author: ducasse $
% $Date: 2009-08-24 10:17:33 +0200 (Mon, 24 Aug 2009) $
% $Revision: 28563 $

% 2011-08-23 - Memento addition Stef chating with dale at ESUG.
% 2011-09-11 - Migrated to PharoBox: svn checkout https://XXX@scm.gforge.inria.fr/svn/pharobooks/PharoByExampleTwo-Eng

%% todo: Stef should go over all the text with the memento in mind and verify all the snippets and stress the right usage in the text


%=================================================================
\ifx\wholebook\relax\else
% --------------------------------------------
% Lulu:
       \documentclass[a4paper,10pt,twoside]{book}
       \usepackage[
              papersize={6.13in,9.21in},
              hmargin={.75in,.75in},
              vmargin={.75in,1in},
              ignoreheadfoot
       ]{geometry}
       \input{../common.tex}
       \setboolean{lulu}{true}
% --------------------------------------------
% A4:
%       \documentclass[a4paper,11pt,twoside]{book}
%       \input{../common.tex}
%       \usepackage{a4wide}
% --------------------------------------------
     \graphicspath{{figures/} {../figures/}}
       \begin{document}
\fi
%=================================================================
%\renewmessage{\nnbb}[2]{} % Disable editorial comments
\sloppy

%=================================================================
%\renewcommand{\nnbb}[2]{#2} % Disable editorial comments


\chapter{Metacello Toolbox: Supporting your development process}
\chalabel{metacello}
\chapterauthor{\authordale{} \\ \authormariano{}}


\section{Script and Tool Support}
Metacello comes with an API to make the writing of tools for Metacello easier. Two classes exist: \ct{MetacelloBaseConfiguration} and \ct{MetacelloToolBox}. 

\subsection{Development Support}

The \ct{MetacelloBaseConfiguration} class is aimed at eventually becoming the common superclass for all Metacello configurations. For now, though, the class serves as the location for defining the common default symbolic versions (\ct{bleedingEdge} at the present time) and as the place to find development support methods such as the following ones:

\begin{description}
\item \ct{compareVersions}: Compare the \ct{#stable} version to \ct{#development} version.
\item \ct{createNewBaselineVersion}: Create a new baseline version based upon the \ct{#stable} version as a model.
\item \ct{createNewDevelopmentVersion}: Create a new \ct{#development} version using the \ct{#stable} version as model.
\item \ct{releaseDevelopmentVersion}: Release \ct{#development} version: set version blessing to \ct{#release}, update the \ct{#development} and \ct{#stable} symbolic version methods and save the configuration.

\item \ct{saveConfiguration}: Save the mcz file that contains the configuration to it's repository.

\item \ct{saveModifiedPackagesAndConfiguration}: Save modified mcz files, update the \ct{#development} version and then save the configuration.

\item \ct{updateToLatestPackageVersions}: Update the \ct{#development} version to match currently loaded mcz files.

\item \ct{validate} Check the configuration for Errors, Critical Warnings, and Warnings.
\end{description}

\subsection{Metacello Toolbox API}

The \ct{MetacelloToolBox} class is aimed at providing a common API for development scripts and Metacello tools. The development support methods were implemented using the Metacello Toolbox API and the OB-Metacello tools have been reimplemented to use the Metacello Toolbox API.

\mmp{did we mention OB-Metacello before?}

For an overview of the Metacello Toolbox API, you can look in the HelpBrowser at the 'Metacello>>API Documentation' section. The instance-side methods of MetacelloToolBox support the  programmatic editing of Metacello configurations from the creation of a new configuration class to the creation and changing of literal and symbolic version methods.

The instance-side methods are intended for the use of tools developers and are covered in the ProfStef tutorial: 'Inside Metacello Toolbox API'. The class-side methods for MetacelloToolBox support a number of configuration management tasks. The target of the initial release of the Metacello Toolbox API is to support the basic Metacello development cycle. In addition to the following section the Metacello development cycle is covered in the ProfStef tutorial: 'Metacello Development Cycle'.


\section{Development Cycle Walk Through}

In this section we'll take a walk through a typical development cycle and provide examples of how the Metacello Toolbox API can be used to support your development process:

\subsection{Example Setup}

When you are developing your project and building your configuration for the first time, you already have the packages that make up your project loaded and correctly running on your image. In this example, it is necessary to load a set of packages to simulate a image that will be used to build the first configuration of the project. We'll cheat here an use an existing configuration (ConfigurationOfGemTools) to download and install in our image all the packages and dependencies needed (just as we would have to do by hand if we were the maintainers of the project). So, don't pay much attention to this step and only focus on the fact that after evaluating it, you'll have loaded in your image all the packages needed to build the example configuration:

\begin{code}{}
Gofer new
  squeaksource: 'MetacelloRepository';
  package: 'ConfigurationOfGemTools';
  load.
((Smalltalk at: #ConfigurationOfGemTools) project version: '1.0-beta.8.3')
  load: 'ALL'.
\end{code}

GemTools is expected to work in Squeak (Squeak3.10 and Squeak4.1) and Pharo (Pharo1.0 and Pharo1.1). GemTools itself is made up of 5 mcz files from the \url{http://seaside.gemstone.com/ss/GLASSClient} repository and  depends upon 4 other projects: FFI, OmniBrowser, Shout and HelpSystem. 
\begin{itemize}
\item OB-SUnitGUI: requires 'OmniBrowser'. 
\item GemTools-Client: requires 'OmniBrowser', 'FFI', 'Shout', and 'OB-SUnitGUI'. 
\item GemTools-Platform: requires 'GemTools-Client'. 
\item GemTools-Help: requires 'HelpSystem' and 'GemTools-Client'. 
\end{itemize}


\subsection{Project Startup}

\subsubsection{Create Configuration and Initial Baseline}
Here we use the toolbox API to create the initial baseline version by specifying the name, repository, projects, packages, dependencyMap and group composition:

\begin{code}{}
  MetacelloToolBox
     createBaseline: '1.0-baseline'
     for: 'GemToolsExample'
     repository: 'http://seaside.gemstone.com/ss/GLASSClient'
     requiredProjects: #('FFI' 'OmniBrowser' 'Shout' 'HelpSystem')
     packages: #('OB-SUnitGUI' 'GemTools-Client' 'GemTools-Platform' 'GemTools-Help' )
     dependencies:
        {('OB-SUnitGUI' -> #('OmniBrowser')).
        ('GemTools-Client' -> #('OmniBrowser' 'FFI' 'Shout' 'OB-SUnitGUI')).
        ('GemTools-Platform' -> #('GemTools-Client')).
        ('GemTools-Help' -> #( 'HelpSystem' 'GemTools-Client'))}
     groups:
        {('default' -> #('OB-SUnitGUI' 'GemTools-Client' 'GemTools-Platform' 'GemTools-Help'))}.
\end{code}                
                
The \ct{createBaseline:...} message copies the class \ct{MetacelloConfigTemplate} to \ct{ConfigurationOfGemToolsExample} (this class is called like this because we have specified in the parameter \ct{for:} that the name of the project was GemToolsExample)  and creates a \ct{#baseline10:} method that looks like the following:

\begin{code}{}
ConfigurationOfGemToolsExample>>baseline10: spec
  <version: '1.0-baseline'>
  spec for: #common do: [
     spec blessing: #'baseline'.
     spec repository: 'http://seaside.gemstone.com/ss/GLASSClient'.
     spec
        project: 'FFI' with: [
          spec
             className: 'ConfigurationOfFFI';
             versionString: #bleedingEdge;
             repository: 'http://www.squeaksource.com/MetacelloRepository' ];
        project: 'OmniBrowser' with: [
          spec
             className: 'ConfigurationOfOmniBrowser';
             versionString: #stable;
             repository: 'http://www.squeaksource.com/MetacelloRepository' ];
        project: 'Shout' with: [
          spec
             className: 'ConfigurationOfShout';
             versionString: #stable;
             repository: 'http://www.squeaksource.com/MetacelloRepository' ];
        project: 'HelpSystem' with: [
          spec
             className: 'ConfigurationOfHelpSystem';
             versionString: #stable;
             repository: 'http://www.squeaksource.com/MetacelloRepository' ].
     spec
        package: 'OB-SUnitGUI' with: [
          spec requires: #('OmniBrowser'). ];
        package: 'GemTools-Client' with: [
          spec requires: #('OmniBrowser' 'FFI' 'Shout' 'OB-SUnitGUI'). ];
        package: 'GemTools-Platform' with: [
          spec requires: #('GemTools-Client'). ];
        package: 'GemTools-Help' with: [
          spec requires: #('HelpSystem' 'GemTools-Client'). ].
     spec group: 'default' with: #('OB-SUnitGUI' 'GemTools-Client'
             'GemTools-Platform' 'GemTools-Help'). ].
\end{code}             



Note that for the 'FFI' project the versionString is \ct{#bleedingEdge}, while the versionString for the other projects is \ct{#stable}. At the time of this writing the FFI project did not have a \ct{#stable} symbolic version defined, so the default versionString is set to \ct{#bleedingEdge}. If a \ct{#stable} symbolic version is defined for the project, then the default versionString is \ct{#stable}. There are no special version dependencies for the GemTools project so the defaults will work just fine. 

%Notice also that it puts the MetacelloRepository by default and this is because of the stablished conventions. Nonetheless, you can change it if there are projects whose configuration classes are not there but in a different repository. In addition, since it does not specify \ct{file:} for every project reference, it also expects that the configuration class is placed in a  package with the same name (a convention we have mentioned different times in the chapter).



\subsection{Create Initial Literal Version}

Now we use the toolbox API to create the initial literal version of the project (by literal we mean with numbers identifying the package versions). The toolbox method \ct{createDevelopment:...} bases the definition of the literal version on the baseline version that we created above and uses the currently loaded state of the image to define the project versions and mcz file versions:

\begin{code}{}
  MetacelloToolBox
     createDevelopment: '1.0'
     for: 'GemToolsExample'
     importFromBaseline: '1.0-baseline'
     description: 'initial development version'.
\end{code}

The \ct{createDevelopment:...} method creates a \ct{#version10:} method in your configuration that looks like this:

\begin{code}{}
ConfigurationOfGemToolsExample>>version10: spec
  <version: '1.0' imports: #('1.0-baseline' )>
  spec for: #common do: [
     spec blessing: #development.
     spec description: 'initial development version'.
     spec author: 'dkh'.
     spec timestamp: '1/12/2011 12:29'.
     spec 
        project: 'FFI' with: '1.2';
        project: 'OmniBrowser' with: #stable;
        project: 'Shout' with: #stable;
        project: 'HelpSystem' with: #stable.
     spec
        package: 'OB-SUnitGUI' with: 'OB-SUnitGUI-dkh.52';
        package: 'GemTools-Client' with: 'GemTools-Client-NorbertHartl.544';
        package: 'GemTools-Platform' with: 'GemTools-Platform.pharo10beta-dkh.5';
        package: 'GemTools-Help' with: 'GemTools-Help-DaleHenrichs.24'. ].
\end{code}

Note how the \ct{#stable} symbolic version specifications were carried through into the literal version. If the version isn't \ct{#stable}, then the currentVersion of the project is filled in, just as the current version of each mcz file is set for the packages. Note also that the blessing of the version '1.0' is set to \ct{#development}. By setting the blessing of a newly created version to \ct{#development}, you indicate that the version is under development and is subject to change without notice. The \ct{createDevelopment:...} method also creates a \ct{#development:} method and specifies that version '1.0' is a \ct{#development symbolic version}:

\begin{code}{}
ConfigurationOfGemToolsExample>>development: spec
  <symbolicVersion: #development>
  spec for: #common version: '1.0'.
\end{code}


\subsection{Validation}

Whenever you finish editing a configuration you should validate it to check for mistakes that may cause problems later on. The Metacello ToolBox provides validations via the message \ct{validateConfiguration:}. The following expression show you possible errors: \ct{(MetacelloToolBox validateConfiguration: ConfigurationOfGemToolsExample) explore}

If the list comes back empty then you are clean. Otherwise you should address the validation issues that show up. Validation issues are divided into three categories:

\begin{description}
\item[Warning -] issues that point out oddities in the definition of a version that do not affect behavior.
\item[Critical Warning -] issues that identify inconsistencies in the definition of a version that may result in unexpected behavior.
\item[ Error -] issues that identify explicit problems in the definition of a version that will result in errors if an attempt is made to resolve the version.
\end{description}

Here's an example of a Critical Warning validation issue:
\begin{code}{}
Critical Warning: No version specified for the project reference 'OCompletion'
                in version '1.1'
     { noVersionSpecified }
     [ ConfigurationOfOmniBrowser #validateVersionSpec: ]
\end{code}     

The first and second line is the explanation, a human readable error message. The third line is the reasonCode, a symbol that represents the category of the issue. You can check out the meanings of the various reasonCodes online or through the following toolbox message:
\ct{(MetacelloToolBox descriptionForValidationReasonCode: #noVersionSpecified)  inspect.}

The fourth line lists the \ct{configurationClass}, \ie the configuration that spawned the issue (there is a different toolbox method for running a recursive configuration validation) and the \ct{callSite}, which is the name of the validation method that generated the error (this is used mainly for debugging).



\subsection{Save Initial Configuration}

The first time you save your configuration, you have to decide where to keep your configuration. It makes sense to keep the configuration in your development repository. The first time that you save your configuration you need to use the MonticelloBrowser or an expression like the following:

\begin{code}{}
  Gofer new
     url: 'http://www.example.com/GemToolsRepository';
     package: 'ConfigurationOfGemToolsExample';
     commit: 'Initial version of configuration'.
\end{code}     


\subsection{Development Cycle}
Now let us look at a typical iteration: testing, releasing, and saving the configuration.
 
\subsubsection{Platform Testing}
To finish the validation of your configuration, you need to do some test loads on your intended platforms. For GemTools we can do a test load into a fresh image (each of the supported PharoCore and Squeak4.1) with the following load expression:

\begin{code}{}
Gofer new
  url: 'http://www.example.com/GemToolsRepository';
  package: 'ConfigurationOfGemToolsExample';
  load.
((Smalltalk at: #ConfigurationOfGemToolsExample)
     project version: #development) load.
\end{code}     
     
Now this is the moment to run the unit tests. Note that for the GemTools unit tests you need to have GemStone installed.

\subsubsection{Release}
Once you are sure that your configuration loads correctly on your target platforms, and your project has reached the desired status, you can release the \ct{#development} into production using the following expression:

\begin{code}{}
  MetacelloToolBox
     releaseDevelopmentVersionIn: ConfigurationOfGemToolsExample
     description: '- release version 1.0'.
\end{code}     

The toolbox method \ct{releaseDevelopmentVersionIn:description:} does the following:
\begin{itemize}
\item set the blessing of the \ct{#development} version to \ct{#release}.
\item sets the \ct{#development} version to \ct{#notDefined}.
\item sets the \ct{#stable} version to the literal version of the \ct{#development} version (in this case '1.0')
\item saves the configuration mcz file to the correct repository.
\end{itemize}


The \ct{development:} method ends up looking like this:

\begin{code}{}
ConfigurationOfGemToolsExample>>development: spec
  <symbolicVersion: #development>
  spec for: #common version: #'notDefined'.
\end{code}  
  
The \ct{stable:} method ends up looking like this:

\begin{code}{}
ConfigurationOfGemToolsExample>>stable: spec
  <symbolicVersion: #stable>
  spec for: #common version: '1.0'.
\end{code}  
  
Finally you can copy the configuration to the MetacelloRepository using the following expression:

\begin{code}{}
  MetacelloToolBox
     copyConfiguration: ConfigurationOfGemToolsExample
     to: 'http://www.squeaksource.com/MetacelloRepository'.
\end{code}


\subsection{Open New Version for Development}

Now we are ready to start new development. The method \ct{createNewDevelopmentVersionIn:...}
performs the necessary modification to be in a state that reflects it. 


\begin{code}{}
MetacelloToolBox
  createNewDevelopmentVersionIn: ConfigurationOfGemToolsExample
  description: '- open 1.1 for development'.
\end{code}

\subsubsection{Configuration Checkpoints}

During the course of development it makes sense to save checkpoints of your development to your repository. To setup this example you should load a newer version of GemTools and get some new mcz files loaded to simulate development:

\begin{code}{}
(ConfigurationOfGemTools project version: '1.0-beta.8.4')
  load: 'ALL'.
\end{code}  

Now that you've simulated some development you can update the \ct{#development} version of your project so that it references the new mcz files you've loaded.

\begin{code}{}
MetacelloToolBox
  updateToLatestPackageVersionsIn: ConfigurationOfGemToolsExample
  description: '- fixed Issue 1090'.
 \end{code} 
 
Then save the configuration to your repository:

\begin{code}{}
MetacelloToolBox
  saveConfigurationPackageFor: 'GemToolsExample'
  description: '- fixed Issue 1090'.
 \end{code}
    
Or do both steps with one toolbox method:

\begin{code}{}
MetacelloToolBox
  saveModifiedPackagesAndConfigurationIn: ConfigurationOfGemToolsExample
  description: '- fixed Issue 1090'.
\end{code}
  
\subsection{Update Project Structure}

In the course of development it is sometimes necessary to add a new package or project. In this case let's add a package to the project called 'GemTools-Overrides' ('GemTools-Overrides' is actually part of the GemTools project already and we just left it out of the previous example). To add a new package to a project you need to:

\begin{itemize}
\item (1) create a new baseline version to reflect the new package and dependencies, 
\item (2) update existing \ct{#development} version to reference the new baseline version, and 
\item (3) include the explicit version for the new package.
\end{itemize}

You can do that manually by using a class browser to manually
copy and edit the old baseline version to reflect the new structure
edit the existing \ct{#development} version method. 
You can also use the Metacello Toolbox API:

\begin{code}{}
| toolbox |
toolbox := MetacelloToolBox configurationNamed: 'GemToolsExample'.
toolbox
  createVersionMethod: 'baseline11:' inCategory: 'baselines' forVersion: '1.1-baseline';
  addSectionsFrom: '1.0-baseline'
     forBaseline: true
     updateProjects: false
     updatePackages: false
     versionSpecsDo: [ :attribute :versionSpec |
        attribute == #common
          ifTrue: [ versionSpec packages add: (toolbox createPackageSpec: 'GemTools-Overrides') ].
        true ];
  commitMethod;
  modifyVersionMethodForVersion: '1.1'
     versionSpecsDo: [ :attribute :versionSpec |
        attribute == #common
          ifTrue: [ versionSpec packages add: (toolbox createPackageSpec: 'GemTools-Overrides') ].
        false ];
  commitMethod.
\end{code}


\mmp{this looks complicated and difficult to understand what he really wants to do.}


\section{Conclusion}
Metacello is an important part of Pharo. It will allow your project to scale. It allows you to control when you want to migrate to new version and for which packages. It is an important architectural backbone.



\paragraph{Proposed development process.}
Using metacello we suggest the following development steps.


\begin{code}{}
Baseline						"first we define a baseline"
Version development			"Then a version tagged as development"
Validate the map				"Once it is validated and the project status arrives to the desired status"
Version release				"We are ready to tag a version as release"	

Version development			"Since development continue we create a new version"
...							"Tagged as development. It will be tagged as release and so on"
Baseline 					"When architecture or structure changes, a new baseline will appear"
Version development			"and the story will continue"
Version release

\end{code}

%=========================================================
\ifx\wholebook\relax\else
    \bibliographystyle{jurabib}
    \nobibliography{scg}
    \end{document}
\fi
%=========================================================



%%% Local Variables: 
%%% coding: utf-8
%%% mode: latex
%%% TeX-master: Lint.tex
%%% TeX-PDF-mode: t
%%% End:
