% $Id$
% % % % % % % % % % % % % % % % % % % % % % % % % % % % % % % % % %
\documentclass[10pt,twocolumn]{article}

% packages
\usepackage{times}
\usepackage{xspace}
\usepackage{ifthen}
\usepackage{amsbsy}
\usepackage{latex8}
\usepackage{balance}
\usepackage{graphicx}
\usepackage{amssymb}

% constants
\newcommand{\Title}{Domain-Specific Program Checking}
\newcommand{\Authors}{Lukas Renggli, Tudor G\^irba, Oscar Nierstrasz}

% references
\usepackage[colorlinks]{hyperref}
\usepackage[all]{hypcap}
\setcounter{tocdepth}{2}
\hypersetup{
	colorlinks=true,
	urlcolor=black,
	linkcolor=black,
	citecolor=black,
	plainpages=false,
	bookmarksopen=true,
	pdfauthor={\Authors},
	pdftitle={\Title}}

% source code
\usepackage{textcomp}
\usepackage{listings}
\usepackage[usenames,dvipsnames]{color}
\definecolor{source}{gray}{0.95}
\lstset{
	language={},
	% characters
	tabsize=3,
	upquote=true,
	escapechar={!},
	keepspaces=true,
	breaklines=true,
	alsoletter={\#:},
	breakautoindent=true,
	columns=fullflexible,
	showstringspaces=false,
	basicstyle=\footnotesize\sffamily,
	% background
	frame=single,
    framerule=0pt,
	backgroundcolor=\color{source},
	% numbering
	numbersep=5pt,
	numberstyle=\tiny,
	numberfirstline=true,
	% captioning
	captionpos=b,
	belowskip=0pt,
	% highlighting
	% moredelim=[s][\color{Green}]{"}{"},
	% moredelim=[s][\color{Purple}]{'}{'},
	% formatting (html)
	moredelim=[is][\textbf]{<b>}{</b>},
	moredelim=[is][\textit]{<i>}{</i>},
	moredelim=[is][\color{Red}\uwave]{<u>}{</u>},
    % characters
    literate=
        {~}{{\raisebox{-1.5ex}{\Large\~{}}}}1
        {-}{{\sf -\hspace{-0.13em}-}}1
        {+}{{\raisebox{0.08ex}{+}}}1
		{>>}{\raisebox{0.3ex}{\hspace{0.1em}\tiny$\boldsymbol{\gg}$\hspace{0.2em}}}1,
}
\newcommand{\ct}{\lstinline[backgroundcolor=\color{white}]}

% proof-reading
\usepackage{xcolor}
\usepackage[normalem]{ulem}
\newcommand{\ra}{$\rightarrow$}
\newcommand{\ugh}[1]{\textcolor{red}{\uwave{#1}}} % please rephrase
\newcommand{\ins}[1]{\textcolor{blue}{\uline{#1}}} % please insert
\newcommand{\del}[1]{\textcolor{red}{\sout{#1}}} % please delete
\newcommand{\chg}[2]{\textcolor{red}{\sout{#1}}{\ra}\textcolor{blue}{\uline{#2}}} % please change
\newcommand{\chk}[1]{\textcolor{ForestGreen}{#1}} % changed, please check

% comments
\newboolean{showcomments}
\setboolean{showcomments}{false}
\ifthenelse{\boolean{showcomments}}
	{\newcommand{\nb}[2]{
		{\colorbox{orange}{\bfseries\sffamily\scriptsize\textcolor{white}{#1}}}
		{\textcolor{orange}{\sf\small$\blacktriangleright$\textit{#2}$\blacktriangleleft$}}}
	 \newcommand{\version}{\emph{\scriptsize$-$Id$-$}}}
	{\newcommand{\nb}[2]{}
	 \newcommand{\version}{}}
\newcommand{\on}[1]{\nb{ON}{#1}}
\newcommand{\lr}[1]{\nb{LR}{#1}}
\newcommand{\TG}[1]{\nb{TG}{#1}}

% graphics: \fig{position}{percentage-width}{filename}{caption}
\DeclareGraphicsExtensions{.png,.jpg,.pdf,.eps,.gif}
\graphicspath{{figures/}}
\newcommand{\fig}[4]{
	\begin{figure}[#1]
		\centering
		\includegraphics[scale=#2]{#3}
		\caption{\label{fig:#3}#4}
	\end{figure}}
\newcommand{\figtwo}[4]{
	\begin{figure*}[#1]
		\centering
		\includegraphics[scale=#2]{#3}
		\caption{\label{fig:#3}#4}
	\end{figure*}}

% abbreviations
\newcommand{\ie}{\emph{i.e.,}\xspace}
\newcommand{\eg}{\emph{e.g.,}\xspace}
\newcommand{\etc}{\emph{etc.}\xspace}
\newcommand{\etal}{\emph{et al.}\xspace}

\newcommand{\Slime}{\textsc{Slime}\xspace}
\newcommand{\Cmsbox}{Cmsbox\xspace}
\newcommand{\Seaside}{Seaside\xspace}
\newcommand{\Helvetia}{\textsc{Helvetia}\xspace}
\newcommand{\Smalltalk}{Smalltalk\xspace}

% references
\newcommand{\tabref}[1]{\hyperref[tab:#1]{Table~\ref{tab:#1}}}
\newcommand{\figref}[1]{\hyperref[fig:#1]{Figure~\ref{fig:#1}}}
\newcommand{\lstref}[1]{\hyperref[lst:#1]{Listing~\ref{lst:#1}}}
\newcommand{\secref}[1]{\hyperref[sec:#1]{Section~\ref{sec:#1}}}
\newcommand{\secreff}[1]{\hyperref[sec:#1]{\ref{sec:#1}~\nameref{sec:#1}}}

% lists
\newenvironment{bullets}[0]
	{\begin{itemize}
		\setlength{\itemsep}{1pt}
		\setlength{\parskip}{0pt}
		\setlength{\parsep}{0pt}}
	{\end{itemize}}

% D O C U M E N T
% % % % % % % % % % % % % % % % % % % % % % % % % % % % % % % % % %
\begin{document}

% T I T L E
% % % % % % % % % % % % % % % % % % % % % % % % % % % % % % % % % %
\title{\Title}
\author{\Authors\vspace{1ex}\\
	Software Composition Group, University of Bern, Switzerland\\
	\url{http://scg.unibe.ch}\\
	\version}
\maketitle
\thispagestyle{empty}

% A B S T R A C T
% % % % % % % % % % % % % % % % % % % % % % % % % % % % % % % % % %

\begin{abstract}
	Lint-like program checkers are popular tools that ensure code quality by verifying compliance with best practices for a particular programming language. The proliferation of internal domain-specific languages, however, poses new challenges for such tools. Traditional program checkers produce many false positives and cannot check constraints, best practices, common errors, possible optimizations and portability issues particular to these domain-specific languages. We advocate the use of dedicated rules to check domain-specific practices. In this paper we report on our experience of maintaining the \Seaside web application framework using such rules. \Seaside defines several internal domain-specific languages that the framework itself and its users depend on. We show how domain-specific program checking significantly improves code quality of \Seaside applications over general purpose program checking.
\end{abstract}

\vspace{-1cm}

\paragraph{Keywords:} Program checking, Program transformation and refactoring, Quality assurance, Domain-Specific Languages

\vspace{1cm}

% % % % % % % % % % % % % % % % % % % % % % % % % % % % % % % % % %
\section{Introduction}\label{sec:introduction}

The use of automatic program checkers to statically locate possible bugs and other problems in source code has a long history. While the first program checkers were part of the compiler, later on separate tools were written that performed a more sophisticated analysis of code to detect possible problem patterns \cite{John83a}. The refactoring book \cite{Fowl99b} made code smell detection popular, as a good indicator to decide when and what to refactor.

Most modern development environments (IDEs) directly provide lint-like tools as part of their editors to warn developers about emerging problems in their source code. These checkers usually highlight offending code snippets on-the-fly and greatly enhance the quality of the written code. Contrary to a separate tool, IDEs with integrated program checkers encourage developers to write good code right from the beginning. Today's program checkers \cite{Hove04a} reliably detect issues like possible bugs, portability issues, violations of coding conventions, duplicated, dead, or suboptimal code, \etc

While a typical program checker works at the level of source code, tools like \emph{intensional views} \cite{Mens06a} and \emph{reflexion models} \cite{Murp95a} have abstracted from the source code and check for structural irregularities and for conformance at an architectural level. % \TG{We have to delimit ourselves here} \lr{that's just an introduction}

Many software projects today use \emph{domain-specific languages} (DSLs) to ease the development process of a particular problem domain. A common approach is to derive a new pseudo-language from an existing API. This technique is known as a \emph{Fluent Interface}, a form of an \emph{internal DSL} or \emph{embedded language} \cite{Fowl08X}. Such languages are syntactically compatible with the host language, and use the same compiler and the same runtime infrastructure.

As internal DSLs often make creative use of host language features, traditional program checkers get confused and produce many false positives. For example, chains of method invocations are normally considered bad practice as they expose internal implementation details and violate the Law of Demeter \cite{Lieb89b}. However, in a fluent interface method chains are often used to formulate domain-specific concerns as a sequence of calls on a single object. As a consequence, traditional program checkers produce false positives and furthermore might miss other issues that are important in the context of the DSL.

In this paper we advocate the use of dedicated rules for internal domain-specific languages. As a running example we use \Seaside\footnote{\url{http://www.seaside.st/}}, an open-source web application framework \cite{Duca07a} written in \Smalltalk. \Seaside defines various internal DSLs to configure application settings, nest components, define the flow of pages, and generate XHTML. As part of his work as \Seaside maintainer and as software consultants on various industrial \Seaside projects, the first author wrote \Slime a Seaside-specific program checker consisting of a set of 30 rules working on the level of the abstract syntax tree (AST). We analyze the impact of these rules on a long term evolution of \Seaside itself and of applications built on top of it.

The paper is structured as follows: \secref{slime} introduces \Slime, a domain-specific program checker for \Seaside code. We present the different domain-specific rules and we discuss how we implemented and integrated the rules. In \secref{case} we report on our experience of applying \Slime rules to various versions of \Seaside and a commercial system that uses \Seaside. \secref{related} discusses related work and \secref{conclusion} concludes.

% % % % % % % % % % % % % % % % % % % % % % % % % % % % % % % % % %
\section{\Slime, domain-specific rules for \Seaside}\label{sec:slime}

The most prominent use of an internal DSL in \Seaside is the generation of HTML. Contrary to most other web application frameworks that use a templating system, in \Seaside HTML is generated programmatically. This DSL is built around a stream-like object that understands messages to conveniently create different XHTML tags. Furthermore the tag objects understand messages to specify the HTML attributes of the generated markup. These attributes are specified using a chain of message sends, known in Smalltalk jargon as a cascade\footnotemark:

\footnotetext{Readers unfamiliar with the syntax of Smalltalk might want to read the code examples aloud and interpret them as normal sentences.

An invocation to a method named \ct{method:with:}, using two arguments looks like: \ct{receiver method: arg1 with: arg2}. The semicolon separates messages that are sent to the same receiver. For example, \ct{receiver method1: arg1; method2: arg2} sends the messages \ct{method1:} and \ct{method2:} to \ct{receiver}. Other syntactic elements of Smalltalk are: the dot to separate statements: \ct{statement1. statement2}; square brackets to denote code blocks or anonymous functions: \ct{[ statements ]}; and single quotes to delimit strings: \ct{'a string'}. The caret \ct{^} returns the result of the following expression.}

\begin{lstlisting}[numbers=left]
html div
	class: 'large';
	with: count.
html anchor
	callback: [ count := count + 1 ];
	with: 'increment'.
\end{lstlisting}

\noindent The above code creates the following HTML markup:

\begin{lstlisting}
<div class="large">0</div>
<a src="/?_s=28hVYPUhdMM7mU&1">increment</a>
\end{lstlisting}

Lines 1--3 are responsible for the generation of the \ct{div} tag with the CSS class \ct{large} and the value of the current \ct{count} as the contents of the tag. Lines 4--6 generate the link with the label \emph{increment}. The \ct{src} attribute is provided by \Seaside. Clicking the link automatically evaluates the code on line 5 and redisplays the component.

This \emph{little language} \cite{Deur97a} for HTML generation is the most prominent use of a DSL in \Seaside. It lets developers abstract common HTML patterns into convenient methods rather than pasting the same sequence of tags into templates every time. While there are other examples of DSLs in \Seaside, we focus on this language only in the remainder of this paper.

As maintainers and users of \Seaside we have observed that while the HTML generation DSL is simple, there are a few common problems that repeatedly appear in the source code. We have collected these problems and categorized them into 4 groups: possible bugs, non-portable code, bad style, and suboptimal code. Spotting such problems early in the development cycle can significantly improve the code quality, maintainability and might avoid hard to detect bugs.

% - - - - - - - - - - - - - - - - - - - - - - - - - - - - - -
\subsection{Possible Bugs}\label{sec:slime-possiblebugs}

This group of rules detects severe problems that are most certainly serious bugs in the source code:

\begin{bullets}
	\item The message \ct{with:} is not last in the cascade,
	\item Instantiates new component while generating HTML, 
	\item Manually invokes \ct{renderContentOn:},
	\item Uses the wrong output stream, 
	\item Misses call to super implementation,
	\item Calls functionality not available while generating output, and 
	\item Calls functionality not available within a framework callback.
\end{bullets}

\paragraph{Example.} As example of such a rule we have a closer look at ``The message \ct{with:} is not last in the cascade''. While in most cases it does not matter in which order the attributes of a HTML tag are specified, \Seaside requires the contents of a tag be specified last. This allows \Seaside to directly stream the tags to the socket, without having to build an intermediate tree of DOM nodes. In the erroneous code below the order is mixed up:

\begin{lstlisting}
html div
	<u>with: count</u>;
	class: 'large'.
\end{lstlisting}

\paragraph{Implementation.} The \Slime rules are implemented on top of \Helvetia\footnote{The implementation along with its source code and examples can be downloaded from \url{http://scg.unibe.ch/research/helvetia}.}, a framework to cleanly extend development tools of the standard Smalltalk IDE. It reuses the existing toolchain of editor, parser, compiler and debugger by leveraging the AST of the host environment. While \Helvetia  is applicable in a much broader context to implement and transparently embed new languages into a host language, in this paper we focus on the program analysis and transformation part of it.

\Slime uses a declarative internal DSL to specify its rules. Every rule is implemented as a method in the class \ct{SlimeRuleDatabase}. \Helvetia automatically collects the result of evaluating these methods to assemble a set of \Slime rules. The following code snippet demonstrates the complete code necessary to implement the rule to check whether \ct{with:} is the last message in the cascade:

\begin{lstlisting}[name=withHasToBeLastInCascade,numbers=left]
<b>SlimeRuleDatabase>>withHasToBeLastInCascade</b>
	^ SlimeRule new
		label: 'The message with: has to be last in the cascade';
		search: (Condition new
			if: [ :context | context isHtmlGeneratingMethod ];
			then: (TreePattern new
				matches: '`html `message with: ``@arguments';
				verification: [ :ast | 
					ast parent isCascade and: [ ast isLastMessage not ] ]));
\end{lstlisting}

Line 2 instantiates the rule object, line 3 assigns a label that appears in the user interface and lines 4--9 define the actual search pattern.

The precondition on line 5 asserts statically that the code artifact under test is used by \Seaside to generate HTML. The \ct{Condition} object lets developers scope rules to relevant parts of the software using the reflective API of the host language. This precondition reduces the number of false positives and greatly increases the performance of the rule.

Line 7 performs a search on the AST for occurrences of statements that follow the pattern \ct{`html `message with: ``@arguments.} The back-ticks in the matching syntax signify variable parts of the expression, which means it does not matter how the variable \ct{`html}, the message \ct{`message:} and the arguments \ct{``@arguments} are exactly named. If a match is found, the AST node is passed into the closure on lines 8 and 9 to verify that the matched node is not the last one of the cascade.

When the \Slime rules are performed by \Helvetia the matching AST nodes are automatically collected. Interested tools can query for these matches and reflect on their type and location in the code. The ``Code Browser'' depicted in \figref{lint} highlights occurrences while editing code. Reporting tools can count, group and sort the issues according to severity.

\fig{h!tb}{0.36}{lint}{Integration of domain-specific rules into the ``Code Browser''.}

\paragraph{Transformation.} Many of the detected problems can be automatically fixed. Providing an automatic refactoring for the above rule is a matter of adding a transformation specification to the above code:

\begin{lstlisting}[name=withHasToBeLastInCascade,numbers=left]
		transform: [ :ast |
			ast cascade
				remove: ast;
				addLast: ast ].
\end{lstlisting}

Line 12 and 13 remove the matched node \ct{ast} from the cascade and re-add it to the end of the sequence. After applying the transformation \Helvetia automatically re-runs the search, to ensure that the transformation actually resolves the problem. 

Again the tools in the IDE automatically offer the possibility to trigger such an automatic transformation. For example, when a developer right-clicks on a \Slime issue in the ``Code Browser'' a confirmation dialog with a preview is presented before the transformation is applied. Furthermore it is possible to ignore and mark false positives, so that they do not show up again.

\Helvetia makes use of the \emph{Rewrite Tool}\footnote{\url{http://www.refactory.com/RefactoringBrowser/Rewrite.html}} that is provided by the Smalltalk refactoring engine \cite{Robe97a} to define and perform searches and transformations on the host language AST. The definition of these patterns is simple, because the host language syntax is used and existing code examples can be generalized by prefixing varying expressions with a back-tick.

% - - - - - - - - - - - - - - - - - - - - - - - - - - - - - -
\subsection{Bad style}\label{sec:slime-badstyle}

These rules detect some less severe problems that might pose maintainability problems in the future but that do not cause immediate bugs. An example of such a rule is ``Extract callback code to separate method''. In the example below the rule proposes to extract the code within the callback into a separate method. This ensures that code related to controller functionality is kept separate from the view.

\begin{lstlisting}
html anchor
	callback: [
		<u>(self confirm: 'Really increment?')</u>
			<u>ifTrue: [ count := count + 1 ]</u> ];
	with: 'increment'.
\end{lstlisting}

\noindent Other rules in this category include:

\begin{bullets}
	\item Use of deprecated API, and
	\item Non-standard object initialization.
\end{bullets}

The implementation of these rules is similar to the one demonstrated in \secref{slime-possiblebugs}.

% - - - - - - - - - - - - - - - - - - - - - - - - - - - - - -
\subsection{Suboptimal Code}\label{sec:slime-suboptimal}

This set of rules suggests optimization that can be applied to code without changing its behavior. For example, the following code triggers the rule ``Unnecessary block passed to brush'':

\begin{lstlisting}
html div with: <u>[ html text: count ]</u>
\end{lstlisting}

The code could be rewritten as follows, but this triggers the rule ``Unnecessary \#with: sent to brush'':

\begin{lstlisting}
html div <u>with: count</u>
\end{lstlisting}

This in turn can be rewritten to the following code which is equivalent to the first version, but much shorter and more efficient as no block closure is activated:

\begin{lstlisting}
html div: count
\end{lstlisting}

% - - - - - - - - - - - - - - - - - - - - - - - - - - - - - -
\subsection{Non-Portable Code}\label{sec:slime-nonportable}

While this set of rules is less important for application code, it is central to the  \Seaside code base itself. The framework runs without modification on 5 different platforms (Squeak, Cincom Smalltalk, GemStone Smalltalk, GNU Smalltalk and Dolphin Smalltalk), which differ in the syntax and the libraries they support. To avoid that contributors from a specific platform accidently submit code that only works on their with their platform we've added some rules that check for compatibility. The rules in this category include:

\begin{bullets}
	\item Invalid object initialization,
	\item Uses curly brace arrays,
	\item Uses literal byte arrays,
	\item Uses method annotations,
	\item Uses non-portable class,
	\item Uses non-portable message,
	\item ANSI booleans,
	\item ANSI collections,
	\item ANSI conditionals,
	\item ANSI convertor,
	\item ANSI exceptions, and
	\item ANSI streams.
\end{bullets}

\paragraph{Example.} Code like \ct{<u>count asString</u>} might not run on all platforms identically, as the convertor method \ct{asString} is not part of the common protocol. Thus, if the code is run on a platform that does not implement \ct{asString} the code might break or produce unexpected results.

\paragraph{Implementation.} The implementation and the automatic refactoring for this issue is particularly simple:

\begin{lstlisting}[numbers=left]
<b>SlimeRuleDatabase>>nonPortableMessage</b>
	^ SlimeRule new
		label: 'Uses non-portable message';
		search: '``@obj asString' replace: '``@obj seasideString';
		search: '``@obj asInteger' replace: '``@obj seasideInteger'
\end{lstlisting}

Again the rule is defined in the class \ct{SlimeRuleDatabase.} It consists of two matching patterns (line 4 and 5 re\-spec\-tive\-ly) and their associated transformation, so code like \ct{count asString} will be transformed to \ct{count seasideString}.

% % % % % % % % % % % % % % % % % % % % % % % % % % % % % % % % % %
\section{Case Studies}\label{sec:case}

Our thesis is that standard program checking tools are not effective when it comes to detecting problems in internal DSLs. In this section we present two case studies where the \Slime rules were applied to control the code quality. The first one is \Seaside itself. The second one is a commercial application based on \Seaside. We analyze several versions of these systems and we compare the results with the number of issues detected by traditional lint rules.

% - - - - - - - - - - - - - - - - - - - - - - - - - - - - - -
\subsection{\Seaside}\label{sec:case-seaside}

\fig{h!bt}{0.65}{seaside}{Average number of Lint and Slime issues per class (above) and lines of code (below) in released Seaside versions.}

\figref{seaside} depicts the average number of issues over various versions of Seaside. The blue line shows the number of standard smells per class (Lint), while the orange line shows the number of domain-specific smells per class (Slime). To give a feel how the size of the code base changes in time, we also display the number of lines of code (LOC) below.

In both cases we observe a significant improvement in code quality between version 2.7 and 2.8. At the time major parts of Seaside were refactored or rewritten to increase portability and extensibility of the code base. No changes are visible for the various 2.8 releases, code quality as measured by the program checkers and lines of code remained constant over time.

Starting with \Seaside 2.9a1 \Slime was introduced in the development process. While the quality as measured by the traditional lint rules remained constant, guiding development by the \Slime rules could significantly improve the quality of the domain-specific code. With the latest version the number of \Slime issues could be significantly reduced.

Feedback we got from early adopters of Seaside 2.9 confirms that the quality of the code is notably better. Especially the portability between different Smalltalk dialects has improved. The code typically compiles and passes the tests directly as it comes from the shared code repository.

An interesting observation is that even if the \Slime smells are reduced and the quality of the code improves, the standard Lint rules continue to report a rather constant portion of problems. This is due to the fact that the generic Lint rules address the wrong level and produce too many false positives.

We further evaluated the number of \emph{false positives} of the remaining open issues in the last analyzed version of Seaside by manually verifying the reported issues: this is 67\% (940 false positives out of 1403 issues reported) in the case of Lint, and 24\% (12 false positives out of 51 issues reported) in the case of \Slime. This demonstrates, that applying dedicated rules provides a better report on the quality of the software than when using the generic rules.

% - - - - - - - - - - - - - - - - - - - - - - - - - - - - - -
\subsection{\Cmsbox}\label{sec:case-cmsbox}

\figtwo{g!tb}{0.65}{cmsbox}{Average number of Lint and Slime issues per class (above) and lines of code (below) in subsequent development versions of the \Cmsbox.}

The \Cmsbox\footnote{\url{http://www.cmsbox.com/}} is a commercial web content management system written in Seaside. \figref{cmsbox} depicts the development of the system over three years. We run the same set of lint and \Slime tests on every fifth version committed, in total over 220 distinct versions were analyzed. Again the lines of code are displayed below, the absolute numbers have been removed to anonymize the data.

In the beginning we observe a rapid increase of detected issues. This is during the initial development phase of the project, where a lot of code was added in a relatively short time. The abrupt drop of lint (and to some smaller extent also \Slime) issues at point (a) can be explained by the removal of a big chunk of experimental code no longer in use. Between versions (a) and (b) the code size grew more slowly, and the code quality remained relatively stable.

During the complete development of the \Cmsbox the standard lint rules were run as part of the daily builds. This explains the that the average number of issues per class is lower than in the case of \Seaside. At point (b) \Slime rules were added and run with every build process. This accounts for the drop of \Slime issues. A new development effort after (b) caused an increasing number of lint issues. The better targeted \Slime rules remained stable.

Contrary to the case study with \Seaside, the \Slime issues do not disappear completely. On the one hand this has to do with the fact that the software is not supposed to run on different platforms, thus the rules that check for conformity on that level were not considered by the development team. On the other hand, the developers were not able to spend a significant amount of time on the issues that were harder to fix and that did not cause immediate problems.

\paragraph{} We see our thesis confirmed in the two case studies: While the general purpose lint rules are definitely useful to be applied to any code base, they are not effective enough when used on an internal DSL. Using detected rules decreases the number of false positives and gives much more relevant information on how to avoid bugs and improve the source code.

% % % % % % % % % % % % % % % % % % % % % % % % % % % % % % % % % %
\section{Related Work}\label{sec:related}

There is a wide variety of tools available to find bugs and check for style issues. Rutar \etal give a good comparison of five bug finding tools for Java \cite{Ruta04a}. 

\emph{PMD} is a program checker that comes with a large collection of different rule-sets. Recent releases also included special rules to check for portability with the Android platform, and common Java technologies such as J2EE, JSP, JUnit, \etc As such, PMD provides some domain-specific rule-sets and encourages developers to create new ones. However, PMD and many other program checker are not tightly integrated into the development IDEs, but instead run as a separate tool. In PMD rules are expressed either as XPath queries or using Java code. In either case PMD provides a proprietary AST that is problematic to keep in sync with the latest Java releases. Furthermore reflective information that goes beyond a single file is not available. This is important when rules require more information on the context, such as the code defined in sub- and superclasses.

Other tools such as \emph{FindBugs} \cite{Hove04a} perform their analysis on bytecode. This has the advantage of being fast, but it requires that the code compiles and completely misses the abstractions of the host language. Thus, writing new rules is very difficult (the developer needs to know how language constructs are represented as bytecode), and targeting internal DLSs is hardly possible.

The Smalltalk Refactoring Browser \cite{Robe97a} comes with more than hundred lint rules, that target common bugs and smells in Smalltalk. While these rules perform well on traditional Smalltalk code, there is an increasing number of false positives when applied to domain-specific code. \Helvetia and \Slime are built on top of the same infrastructure. This provides us with excellent tools for introspection and intercession of the AST in the host system, and prevents us from building our own proprietary tools to parse, query and transform source code. \Helvetia adds the rule system to declarative compose conditions, matchers and transformations and to scope and integrate the rules into the existing development tools. 

High-level abstractions can be won by recovering the structural model of the code. \emph{Intensional Views} document structural regularities in source code and check for conformance against various versions of the system \cite{Mens06a}. \emph{Software reflexion models} extract high-level models from the source code and compare them with models the developer has specified. Our approach works on a different dimension, we do not seek for a high-level model of the code but instead stay at the language level of the host language model and embrace the domain-specific knowledge.

% % % % % % % % % % % % % % % % % % % % % % % % % % % % % % % % % %
\section{Conclusion}\label{sec:conclusion}

Our case studies revealed that rules that are targeted at a particular problem domain usually performed better and cause fewer false positives than general purpose lint rules. While more evidence is needed, these initial case studies do point out the benefits of using rules dedicated to internal DSLs over using generic ones.

We argued that in general we need to accommodate such custom detections at the IDE level, and we presented \Slime, a domain-specific program checker built on top of a generic infrastructure for modeling and analyzing languages embedded in a host language.

As we need to accommodate unexpected DSLs, the ability to add new dedicated rules is crucial. In \Slime, adding domain-specific rules is straightforward. Using the \Helvetia framework it is possible to declaratively specify new rules and closely integrate them with the host environment. Rules are scoped to certain parts of the system using the reflective API of the host language. Furthermore the infrastructure of the refactoring tools helped us to efficiently perform searches and transformations on the AST nodes of host system \cite{Denk07b}.

We applied the presented techniques only to internal DSLs, that by definition share the same syntax as the host language. As a generalization we envision to extend this approach to any \emph{embedded language}, that does not necessarily share the same syntax as the host language. \Helvetia uses the AST of the host environment as the common representation of all executable code, thus it is always possible to run the rules at that level. Since \Helvetia automatically keeps track of the source location it is possible to provide highlighting of lint issues in other languages. The challenge however will be to express the rules in terms of the embedded language. This is not only necessary to be able to offer automatic transformations, but also more convenient for rule developers as they do not need to work on two different abstraction levels.

% \begin{table*}[h!tb]
% 	\centering
% 	\begin{tabular}{lr}
% 		\textbf{\Slime Rule} & \textbf{LOC} \\
% 		\hline 
% 		The message with: is not last in the cascade & 13 \\
% 		Instantiates new component while generating HTML & 15 \\
% 		Manually invokes renderContentOn: & 4 \\
% 		Uses the wrong output stream & 16 \\
% 		Misses call to super implementation & 20 \\
% 		Calls functionality not available while generating output & 5 \\
% 		Calls functionality not available within a framework callback & 5 \\
% 		Invalid object initialization & 9 \\
% 		Uses curly brace arrays & 4 \\
% 		Uses literal byte arrays & 4 \\
% 		Uses method annotations & 4 \\
% 		Uses non-portable class & 4 \\
% 		Uses non-portable message & 4 \\
% 		ANSI booleans & 11 \\
% 		ANSI collections & 17 \\
% 		ANSI conditionals & 13 \\
% 		ANSI convertor & 4 \\
% 		ANSI exceptions & 5 \\
% 		ANSI streams & 5 \\
% 		Use of deprecated API & 8 \\
% 		Non-standard object initialization & 9 \\
% 	\end{tabular}
% 	\caption{Comparison}
% \end{table*}

% % % % % % % % % % % % % % % % % % % % % % % % % % % % % % % % % %
\section*{Acknowledgments} \balance

We thank Adrian Lienhard and \href{http://www.nestyle.ch}{nestyle.ch} for providing the Cmsbox case study. We gratefully acknowledge the financial support of the Swiss National Science Foundation for the project ``Bringing Models Closer to Code" (SNF Project No.\ 200020-121594, Oct.\ 2008 -- Sept.\ 2010).

% bibliography
% % % % % % % % % % % % % % % % % % % % % % % % % % % % % % % % %
\bibliographystyle{latex8}
\bibliography{scg}

\end{document}
