% $Author: ducasse $
% $Date: 2009-07-06 11:55:40 +0200 (Mon, 06 Jul 2009) $
% $Revision: 27889 $

% HISTORY:
% 2008-08-19 - Stef started chapter (outline only)
% 2011-09-11 - Migrated to PharoBox: svn checkout https://XXX@scm.gforge.inria.fr/svn/pharobooks/PharoByExampleTwo-Eng

%=================================================================
\ifx\wholebook\relax\else
% --------------------------------------------
% Lulu:
	\documentclass[a4paper,10pt,twoside]{book}
	\usepackage[
		papersize={6.13in,9.21in},
		hmargin={.75in,.75in},
		vmargin={.75in,1in},
		ignoreheadfoot
	]{geometry}
	\input{../common.tex}
	\setboolean{lulu}{true}
% --------------------------------------------
% A4:
%	\documentclass[a4paper,11pt,twoside]{book}
%	\input{../common.tex}
%	\usepackage{a4wide}
% --------------------------------------------
    \graphicspath{{figures/} {../figures/}}
	\begin{document}
\fi
%=================================================================
%\renewcommand{\nnbb}[2]{} % Disable editorial comments
\sloppy
%=================================================================
\chapter{Checking and Transforming Programs with Rewrite rules}\chalabel{rewrite}




\section{Automatic manipulation of Code}




\section{Basic Knowledge on AST and other concepts}


\begin{code}{a Simple method}
< aPoint 
	"Answer whether the receiver is above and to the left of aPoint."

	^x < aPoint x and: [y < aPoint y]
\end{code}
	
	
\begin{code}{Getting the AST of a method}
( Point >> #<) parseTree
\end{code}	


\subsection{Basic on ProgramNodes and Subclasses}
Classes represent different nodes

ProgramNode is the superclass

Implement several useful methods: parent, nodesDo:, isVariable....



\subsection{Unification}
The pattern matchin algorithm in the parser tree rewriter is accomplished by a unification algorithm. This algorithm takes as input a tree (to search) and a pattern (a structure containing pattern variables).

The output of the algorithm is either False (there is no match found) or true and a set of assignments describing the variables. 

The set of assignments contains the values of the pattern variables that make the match between the tree and the pattern variables succeed. 

For example:
\begin{code}{}
self X matches self open if X = 'open'

X Y matches self open if X = 'self' and Y = 'foo'

X X does not match self open.
\end{code}



\section{Getting Started}

\begin{code}{Matching}
| className realClass replacer category |
className := #MyClass.
realClass := Smalltalk at: className.
category := #accessing.

\end{code}





If you really just want the string 'tabs', the string 'tabs' with the
quotes is the search expression.

If you want to find it as part of a substring use something along:
\begin{code}{}
   `#string `{ :node | node value isString and: [ node value
includesSubString: 'tabs' ] }

The `#string is a literal pattern (booleans, characters, arrays,
strings, numbers, ...) and `{ ... adds a constraint on the preceeding
match.
\end{code}



\begin{code}{}
| className realClass replacer category |

className := #MyClass.
realClass := Smalltalk at: className.
category := #accessing.

replacer := RBParseTreeRewriter new
				replace: '`receiver oldMessage' with: '`receiver newMessage';
				yourself.
(realClass organization listAtCategoryNamed: category)
	collect: [:sel |
		| parseTree |
		parseTree := ( realClass >> sel) parseTree.
		(replacer executeTree: parseTree)
			ifTrue: [ realClass compile: replacer tree newSource " [1] " ] ]

Now some questions:

1) In [1] "( RBClass existingNamed: className ) compileTree: replacer
tree" would be more appropiate? I get a MNU when evaluated with latest
AST/Refactoring.
2) The above script have some problems, if you want to get replaced

'this is a string' asString.

to

'this is a string' asText

it won't match. The same happens with: " self model asString " and
other patterns. 

answer>>>`#literal asString


How to modify the rewriter to match
oldMessage/newMessage at any point?

>>>``@obj oldMessage


i.e. [: tmp | anObject1 blabla1 blabla2 oldMessage blabla3 ]

Thanks in advance,
\end{code}


answer from lukas
\begin{code}{}
RBClass is a class private to the framework, you should never need to
instantiate it directly. And if you do, only through an instance of
RBNamespace that provides a delta to the current system state. In your
example you don't need to do that.

\end{code}

%=================================================================
\ifx\wholebook\relax\else\end{document}\fi
%=================================================================

%-----------------------------------------------------------------

%%% Local Variables:
%%% coding: utf-8
%%% mode: latex
%%% TeX-master: t
%%% TeX-PDF-mode: t
%%% ispell-local-dictionary: "english"
%%% End: