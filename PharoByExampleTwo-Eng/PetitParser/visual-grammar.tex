\usetikzlibrary{chains,positioning,matrix,scopes,decorations.shapes,arrows,shapes}


% fuer Railroad-Diagramme
\tikzset{
  production/.style={
    % The shape:
    rectangle,
    % The size:
    minimum size=6mm,
    % The border:
    % very thin,
    % draw=red!50!black!50,         % 50% red and 50% black,
    %                               % and that mixed with 50% white
    % % The filling:
    % top color=white,              % a shading that is white at the top...
    % bottom color=red!50!black!20, % and something else at the bottom
    % Font
    font=\sffamily
  },
  nonterminal/.style={
    % The shape:
    rectangle,
    % The size:
    minimum size=6mm,
    % The border:
    very thick,
    draw=red!50!black!50,         % 50% red and 50% black,
                                  % and that mixed with 50% white
    % The filling:
    top color=white,              % a shading that is white at the top...
    bottom color=red!50!black!20, % and something else at the bottom
    % Font
    font=\ttfamily
  },
  terminal/.style={
    % The shape:
    rounded rectangle,
    minimum size=6mm,
    % The rest
    very thick,draw=black!50,
    top color=white,bottom color=black!20,
    font=\ttfamily},
  skip loop/.style={to path={-- ++(0,#1) -| (\tikztotarget)}}
}
{
  \tikzset{terminal/.append style={text height=1.5ex,text depth=.25ex}}
  \tikzset{nonterminal/.append style={text height=1.5ex,text depth=.25ex}}
}

\newcommand{\tikzgrammar}[1]{%
    \begin{tikzpicture}[point/.style={coordinate},>=stealth',thick,draw=black!50,%
      tip/.style={->,shorten >=0.007pt},every join/.style={},%
      hv path/.style={to path={-| (\tikztotarget)}},%
      vh path/.style={to path={|- (\tikztotarget)}},%
      text height=1.5ex,text depth=.25ex]%
      \matrix[ampersand replacement=\&,column sep=4mm, row sep=4mm]%
      #1
    \end{tikzpicture}%
}

\newcommand{\tikzgrammarfig}[2]{%
  \begin{figure}
    \centering
    \tikzgrammar{#2}
    \caption{Syntax diagram representation for the #1 parser defined in \scrref{#1}}
\label{fig:syntax-#1}
\end{figure}
}

\newcommand{\syntaxref}[1]{\figref{syntax-#1}}