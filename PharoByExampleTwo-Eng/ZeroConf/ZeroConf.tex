% $Author: Benjamin $
% $Date: 2011-12-21 14:28:33 +0200 $
% $Revision: 28563 $

%=================================================================
\ifx\wholebook\relax\else
% --------------------------------------------
% Lulu:
	\documentclass[a4paper,10pt,twoside]{book}
	\usepackage[
		papersize={6.13in,9.21in},
		hmargin={.75in,.75in},
		vmargin={.75in,1in},
		ignoreheadfoot
	]{geometry}
	\input{../common.tex}
	\setboolean{lulu}{true}
% --------------------------------------------
% A4:
%	\documentclass[a4paper,11pt,twoside]{book}
%	\input{../common.tex}
%	\usepackage{a4wide}
% --------------------------------------------
    \graphicspath{{figures/} {./figures/}}
	\begin{document}
\fi
%=================================================================
%\renewmessage{\nnbb}[2]{} % Disable editorial comments
\sloppy
%=================================================================

\chapter{Zero Configuration Scripts and Command-Line Handlers}
\chapterauthor{Camillo Bruni}

Weren't you fed up not be able to install Pharo from a single command line or to pass it arguments? 
Using a nice debugger and an interactive environment development does not 
mean that Pharo developers do not value automatic scripts and love the command line.
Yes we do and we want the best of both worlds! We really wanted it to free our mind of retaining arbitrary information. A zero configuration is a script that automatically downloads everything you need to get started. Since version 2.0, Pharo also supports a way to define and handle command line arguments. 

This chapter shows how to get the zeroconf scripts for Pharo as well as how you can pass arguments to the environment from the command-line.


%:==========
\section{Getting the VM and the Image}
First here is a way to download a zero configuration script to download the latest 2.0 Pharo image and vm. 

\begin{code}{}
wget get.pharo.org/20+vm
\end{code}

If you do not have \ct{wget} installed you can use \ct{curl -L} instead.

To execute the script that we just downloaded, you should change its permissions using \ct{chmod a+x}  or invoke it via bash as follows. 

\paragraph{Configurations.}
There is a plethora of configurations available. The URL for each script can
be easily built from an image version and a vm following the expression: \ct{get.pharo.org/$IMAGE+$VM}
	
Possible values for \ct{$IMAGE} are: \ct{12 13 14 20 30 stable alpha}

Possible values for \ct{$VM} are: \ct{vm vmS vmLatest vmSLatest}

Of course, one can just download an image as well \ct{get.pharo.org/$IMAGE} or just the VM \ct{get.pharo.org/$VM}



\paragraph{Looking at the help.}
Now let's have a look at the script help.

\begin{code}{}
bash 20+vm --help 
\end{code}

The help says that the \ct{20+vm} command downloads the current virtual machine and puts it into the pharo-vm folder. In addition, it creates several scripts: \ct{pharo} to launch the system, \ct{pharo-ui} a script to launch the image in UI mode. Finally it also downloads the latest image and changes files.

\begin{code}{Output of 20+vm --help }
This script downloads the latest Pharo 20 Image.
This script downloads the latest Pharo VM.

The following artifacts are created:
    Pharo.changes  A changes file for the Pharo Image
    Pharo.image    A Pharo image, to be opened with the Pharo VM
    pharo          Script to run the downloaded VM in headless mode
    pharo-ui       Script to run the downloaded VM in UI mode
    pharo-vm/      Directory containing the VM
\end{code}



\paragraph{Grabbing and executing it.}
If you just want to directly execute the script you can also do the following

\begin{code}{}
wget -O - get.pharo.org/20+vm | bash
\end{code}

The option \ct{-O -} will output the downloaded bash file to standard out, so we can pipe it to \mbox{\ct{bash}.} If you do not like the log of web, use  \ct{--quiet}.

\begin{code}{}
wget --quiet -O - get.pharo.org/20+vm | bash
\end{code}



\paragraph{Note for the believers in automated tasks.} 
The scripts are fetched automatically from our jenkins server (\url{https://ci.inria.fr/pharo/job/Scripts-download/}) from the gitorious server \mbox{\url{https://gitorious.org/pharo-build/pharo-build}.}
Yes we believe in automated tasks that free our energy. 

\section{Getting the VM only}
You can also use different scripts. For example \ct{get.pharo.org/vm} only downloads the latest vm.

\begin{code}{}
wget -O - get.pharo.org/vm | bash
\end{code}


Again as any script you can always check its help message.

\begin{code}[]{Output of get.pharo.org/vm --help}
This script downloads the latest Pharo VM.
The following artifacts are created:
    pharo      Script to run the downloaded VM in headless mode
    pharo-ui   Script to run the downloaded VM in UI mode
    pharo-vm/  Directory containing the VM
\end{code}


Figure~\ref{fig:website} shows the list of scripts available that you can get at \mbox{\url{http://get.pharo.org}.}

\begin{figure}[!h]
	\centering
	\includegraphics[width=\textwidth]{zeroconfwebsite}
	\caption{All the scripts are available at \ct{http://get.pharo.org}.\label{fig:website}}
\end{figure}



\section{Handling command line options}
We have now a brand new and nice way to handle command line arguments. It is self documented and easily extendable. Let us have a look at how the command line is handled. As usual we will start to show you how to find your way alone.


\subsection{How to find our way}
Again we love and value self documentation so just use the  \ct{--help} option to get an explanation. 

\begin{code}{}
./pharo Pharo.image --help
\end{code}

It will produce the following output.

\begin{code}{}
Usage: [<subcommand>] [--help] [--copyright] [--version] [--list]
	--help print this help message
	--copyright print the copyrights
	--version print the version for the image and the vm
	--list list a description of all active command line handlers
	<subcommand> a valid subcommand in --list
	
Documentation:
A DefaultCommandLineHandler handles default command line arguments and options.
The DefaultCommandLineHandler is activated before all other handlers. 
It first checks if another handler is available. If so it will activate the found handler.
\end{code}


\subsection{System version and handler list}
Two of the default options are important \ct{versions} and \ct{list}. Let us have a look to them now.



\paragraph{Getting system version.} A typical and important command line option is \ct{--version}. Please use it when you communicate bugs and deviant behavior. 

\begin{code}{}
./pharo Pharo.image --version
M:    NBCoInterpreter NativeBoost-CogPlugin-IgorStasenko.15 uuid: 44b6b681-38f1-4a9e-b6ee-8769b499576a Dec 18 2012
NBCogit NativeBoost-CogPlugin-IgorStasenko.15 uuid: 44b6b681-38f1-4a9e-b6ee-8769b499576a Dec 18 2012
git://gitorious.org/cogvm/blessed.git Commit: 452863bdfba2ba0b188e7b172e9bc597a2caa928 Date: 2012-12-07 16:49:46 +0100 By: Esteban Lorenzano <estebanlm@gmail.com> Jenkins build #5922
\end{code}

The \ct{--version} argument gives the version of the virtual machine. If you wish to obtain the version of the image, then you need to open the image, use the World menu, and select About.

\paragraph{List of available handlers.} The command line option \ct{--list} is interesting since it give you the list of the \emph{current} option handlers. This list depends on the handlers that are currently loaded in the system. In particular it means that you can simply add an handler for your specific situation and wishes.

The following list shows that handlers available in the system we used when writing this chapter.

\begin{code}{}
./pharo Pharo.image --list

Currently installed Command Line Handlers:
    st              Loads and executes .st source files
    Fuel            Loads fuel files
    config          Install and inspect Metacello Configurations from the command line
    save            Rename the image and changes file
    test            A command line test runner
    update          Load updates
    printVersion    Print image version
    eval            Directly evaluates passed in one line scripts
\end{code}

Now you probably wonder how certain option should be used. Indeed it does not look like using the config option which handles the loading of Metacello configurations is crystal clear, isn't. But as you guessed handler are also self described. Let us have a look at this one. 



\paragraph{Loading Metacello Configuration.}
To get some explanation about the use of the config option, just request its associated help as follows: 

\begin{code}{}
./pharo Pharo.image config --help
\end{code}

Note that this help is the one of the associated handler not the one of the command line generic system. 

\begin{code}{}
Usage: config [--help] <repository url> [<configuration>] [--install[=<version>]] [--group=<group>] [--username=<username>] [--password=<password>]
	--help              show this help message
	<repository url>    A Monticello repository name 
	<configuration>     A valid Metacello Configuration name
	<version>           A valid version for the given configuration
	<group>             A valid Metacello group name
	<username>          An optional username to access the configuration's repository
	<password>          An optional password to access the configuration's repository
	
Examples:
	# display this help message
	pharo Pharo.image config
	
	# list all configurations of a repository
	pharo Pharo.image config $MC_REPOS_URL
	
	# list all the available versions of a confgurtation
	pharo Pharo.image config $MC_REPOS_URL ConfigurationOfFoo
	
	# install the stable version
	pharo Pharo.image config $MC_REPOS_URL ConfigurationOfFoo --install
	
	#install a specific version '1.5'
	pharo Pharo.image config $MC_REPOS_URL ConfigurationOfFoo --install=1.5
	
	#install a specific version '1.5' and only a specific group 'Tests'
	pharo Pharo.image config $MC_REPOS_URL ConfigurationOfFoo --install=1.5 --group=Tests
\end{code}
% $ <--- don't remove that, it's here to fix a bug in Emacs


\section{Anatomy of a handler}
As we mentioned it, the command line mechanism is open and can be extended. We will look now how the handler for the eval option is defined. 


\paragraph{Evaluating Pharo Expressions.} You can use the command line to evaluate expressions as follows: \ct{./pharo Pharo.image eval '1+2'}

\begin{code}{}
./pharo Pharo.image eval --help 
Usage: eval [--help] <smalltalk expression>
	--help    list this help message
	<smallltalk expression>  a valid Smalltalk expression which is evaluated and 
	                         the result is printed on stdout

Documentation:
A CommandLineHandler that reads a string from the command line, outputs the evaluated result and quits the image. 

This handler either evaluates the arguments passed to the image:
	$PHARO_VM my.image eval  1 + 2
	
or it can read directly from stdin:

	echo "1+2" | $PHARO_VM my.image eval 
\end{code}

Now the handler is defined as follows: First we define a subclass of CommandLineHandler. Here 
\ct{BasicCodeLoader} is a subclass of \ct{CommandLineHandler} and \ct{EvaluateCommandLineHandler} is a subclass of \mbox{\ct{BasicCodeLoader}.}

\begin{code}{}
BasicCodeLoader subclass: #EvaluateCommandLineHandler
	instanceVariableNames: ''
	classVariableNames: ''
	poolDictionaries: ''
	category: 'System-CommandLine'
\end{code}

Then we define the \ct{commandName} on the class side as well as the method \ct{isResponsibleFor:}.

\begin{code}{}
EvaluateCommandLineHandler class>>commandName
	^ 'eval'

EvaluateCommandLineHandler class>>isResponsibleFor: commandLineArguments
	"directly handle top-level -e and --evaluate options"
	commandLineArguments withFirstArgument: [ :arg| 
		(#('-e' '--evaluate') includes: arg)
			ifTrue: [ ^ true ]].
	
	^ commandLineArguments includesSubCommand: self commandName

EvaluateCommandLineHandler class>>description
	^ 'Directly evaluates passed in one line scripts'
\end{code}

Then we define the method \ct{activate} which will be executed when the option matches. 
\begin{code}{}
EvaluateCommandLineHandler>>activate
	self activateHelp.
	self arguments ifEmpty: [ ^ self evaluateStdIn ].
	self evaluateArguments.
	self quit.
\end{code}

In particular we define a class comment since this is this class comment that will be printed when the help is requested. 

%\begin{figure}[!h]
%	\centering
%	\includegraphics[width=\textwidth]{Jenkins}
%	\caption{XMLWriter ZeroConf scripts. \label{fig:jenkins}}
%\end{figure}

If you want your image saved at the end of an evaluation script, pass the \ct{--save} option just after \ct{eval}.

\section{Using ZeroConf script with Jenkins}
Now that we have such scripts and the possibility to specify option, we can write jenkins scripts which are relying as less as possible on bash. 

For example here is the command that we use in jenkins for the project XMLWriter (which is hosted on PharoExtras). 

\begin{code}{}
# jenkins puts all the params after a / in the job name as well :(
export JOB_NAME=`dirname $JOB_NAME`

wget --quiet -O - get.pharo.org/$PHARO+$VM | bash

./pharo Pharo.image save $JOB_NAME --delete-old
./pharo $JOB_NAME.image --version > version.txt

REPO=http://smalltalkhub.com/mc/PharoExtras/$JOB_NAME/main
./pharo $JOB_NAME.image config $REPO ConfigurationOf$JOB_NAME --install=$VERSION --group='Tests'
./pharo $JOB_NAME.image test --junit-xml-output "XML-Writer-.*"

zip -r $JOB_NAME.zip $JOB_NAME.image $JOB_NAME.changes
\end{code}

\section{Conclusion}
You can now really easily get access to the latest version of Pharo and build scripts. In addition the 
command-line handler opens new horizons to be used in shell scripts.

%=========================================================
\ifx\wholebook\relax\else
   \bibliographystyle{jurabib}
   \nobibliography{scg}
   \end{document}
\fi

%=================================================================
\ifx\wholebook\relax\else\end{document}\fi
%=================================================================

%-----------------------------------------------------------------

%%% Local Variables:
%%% coding: utf-8
%%% mode: latex
%%% TeX-master: t
%%% TeX-PDF-mode: t
%%% ispell-local-dictionary: "english"
%%% End:


%%Example:
%%========
%%The scripts work the same way as before:
%%
%%Load the stable Pharo version with a VM:  wget -qO - get.pharo.org       | bash
%%Load Pharo 3.0 with a VM:                 wget -qO - get.pharo.org/30+vm | bash
%%Load only a Pharo 3.0 image:              wget -qO - get.pharo.org/30    | bash
%%
%%If you do not have wget installed you can use `curl -L` instead of `wget -qO -`.
%%
%%=> All the jobs at http://ci.inria.fr/rmod already use the new scripts
%%
%%Details:
%%========
%%There is a plethora of configurations available. The URL for each script can
%%be easily build from an image version and a vm:
%%	get.pharo.org/$IMAGE +$VM
%%Of course one can just download an image as well
%%	get.pharo.org/$IMAGE
%%or just the VM
%%	get.pharo.org/$VM
%%
%%Possible values for $IMAGE are:           12 13 14 20 30 stable alpha
%%Possible values for $VM are:              vm vmS vmLatest vmSLatest
%%
%%Help:
%%=====
%%I will update the zeroconf chapter tomorrow, otherwise you can simply open the url
%%of the config script in the browser to get a help message that explains you how to
%%use the script and what it downloads.
%%
%%
%%Migration:
%%==========
%%- Replace http://get.pharo.org/ci*.sh with the proper get.pharo.org/$IMAGE+$VM url
%%- Replace ./vm.sh with ./pharo
%%- Replace ./vm-ui.sh with ./pharo-ui

