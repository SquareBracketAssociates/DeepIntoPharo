% $Author:  $
% $Date: 2009-07-06 10:21:47 +0200 (Mon, 06 Jul 2009) $
% $Revision: 27886 $
% HISTORY:

%=================================================================
\ifx\wholebook\relax\else
% --------------------------------------------
% Lulu:
	\documentclass[a4paper,10pt,twoside]{book}
	\usepackage[
		papersize={6.13in,9.21in},
		hmargin={.75in,.75in},
		vmargin={.75in,1in},
		ignoreheadfoot
	]{geometry}
	\input{../common.tex}
	\pagestyle{headings}
	\setboolean{lulu}{true}
% --------------------------------------------
% A4:
%	\documentclass[a4paper,11pt,twoside]{book}
%	\input{../common.tex}
%	\usepackage{a4wide}
% --------------------------------------------
    \graphicspath{{figures/} {../figures/}}
	\begin{document}
	% \renewcommand{\nnbb}[2]{} % Disable editorial comments
	\sloppy
	\frontmatter
\fi
%=================================================================
\chapter{Preface}\chalabel{intro}

Pharo by Example (freely available at \ct{http://pharobyexample.org}) is an important book to help newcomers to feel confident with Pharo. 
The current book, Deep into Pharo, proposes to the reader a deeper travel into important parts of Pharo. It covers new libraries such as FileSystem, frameworks such as Roassal, under documented aspects such as exceptions or blocks. 

The first part of the book covers system aspects of Pharo with FileSystem a new file library, Settings a new architecture to manage preferences, zero configuration scripts, sockets and regular expressions. The second part focuses on the package system of Pharo: it presents Monticello for source code versioning, Gofer to script it and Metacello to build large projects and specify their dependencies. 

%=================================================================
\section*{Acknowledgments}

We would like to thank various people who have contributed to this book. In particular, we would like to thank:
\begin{itemize}
\item Oscar Nierstrasz for writing and co-editing some chapters such as Regex and Monticello.
\item Lukas Renggli for PetitParser and his work on the refactoring engine and smallLint rules. 
\item Colin Putney for the initial version of FileSystem and Camillo Bruni for his review of FileSystem and his rewrite of the Pharo Core.
\item Alain Plantec for the Settings framework and its effort to integrate it into Pharo. 
\item Doru Girba for Glamour and the first documentation.
\end{itemize}

We would like to thanks Hernan Wilkinson and Carlos Ferro for their reviews, Nicolas Cellier for the feedback on the number chapter, and Vassili Bykov for permission to adapt his Regex documentation.

We thank Inria Lille Nord Europe for supporting this open-source project and for hosting the web site of this book.

We also thank the Pharo community for their enthusiastic support of this project, and for informing us of the errors found in the first edition of this book.



%=============================================================
\ifx\wholebook\relax\else
   \bibliographystyle{jurabib}
   \nobibliography{scg}
   \end{document}
\fi
%=============================================================
