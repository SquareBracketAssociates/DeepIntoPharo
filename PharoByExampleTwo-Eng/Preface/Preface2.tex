% $Author:  $
% $Date: 2009-07-06 10:21:47 +0200 (Mon, 06 Jul 2009) $
% $Revision: 27886 $
% HISTORY:

%=================================================================
\ifx\wholebook\relax\else
% --------------------------------------------
% Lulu:
	\documentclass[a4paper,10pt,twoside]{book}
	\usepackage[
		papersize={6.13in,9.21in},
		hmargin={.75in,.75in},
		vmargin={.75in,1in},
		ignoreheadfoot
	]{geometry}
	\input{../common.tex}
	\pagestyle{headings}
	\setboolean{lulu}{true}
% --------------------------------------------
% A4:
%	\documentclass[a4paper,11pt,twoside]{book}
%	\input{../common.tex}
%	\usepackage{a4wide}
% --------------------------------------------
    \graphicspath{{figures/} {../figures/}}
	\begin{document}
	% \renewcommand{\nnbb}[2]{} % Disable editorial comments
	\sloppy
	\frontmatter
\fi
%=================================================================
\chapter{Preface}\chalabel{intro}

\indent

 \begin{quote}
   \emph{``Smalltalk is well known as an excellent tool for agile and exploratory programming.  In this book the authors present a new dialect of Smalltalk called Pharo that has been specifically designed for inventive developers.  The authors are key members of the Pharo team and accomplished OO educators, researchers and designers.   Numerous Smalltalk projects from the authors and others have been ported to Pharo.  Enjoy Deep Into Pharo''} \\ - Dave Thomas\footnote{David (\url{http://www.davethomas.net}) is a well-known figure in modern software development and object technology. Thomas is perhaps best known as the founder and past CEO of Object Technology International, Inc., now IBM OTI Labs. OTI was responsible for initial development of the Eclipse open source IDE and the Visual Age Java development environment.} -   
\end{quote}

Using a programming language is so far the most convenient way for a human to tell  a computer what it should do. Pharo is an object-oriented programming language, highly influenced by Smalltalk. Pharo is more than a syntax and a bunch of semantics rules as most programming languages are. Pharo comes with an extensible and flexible programming environment. Thanks to its numerous object-oriented libraries and frameworks, Pharo shines for modeling and visualizing data, scripting, networking and many other ranges of applications.

The very light syntax and the malleable object model of Pharo are commonly praised. Both early learners and experienced programmers enjoy the ``everything is an object'' paradigm. The simplicity and expressiveness of Pharo as well as a living environment empowers programmers with a wonderful and unique sensation. 

Deep into Pharo is the second volume of a book series initiated with Pharo by Example\footnote{freely available from \ct{http://pharobyexample.org}}. 
Deep into Pharo, the book you are reading, accompanies the reader for a fantastic journey into exciting parts of Pharo. It covers new libraries such as FileSystem, frameworks such as Roassal and Glamour, complex of the system aspects such as exceptions and blocks. 

The book is divided into 5 parts and 17 chapters. The first part deals with truly object-oriented libraries. The second part is about source code management. The third part is about advanced frameworks. The fourth part covers advanced topics of the language, in particular exception, blocks and numbers. The fifth and last part is about tooling, including profiling and parsing.

%The first part of the book covers system aspects of Pharo with FileSystem a new file library, Settings a new architecture to manage preferences, zero configuration scripts, sockets and regular expressions. The second part focuses on the package system of Pharo: it presents Monticello for source code versioning, Gofer to script it and Metacello to build large projects and specify their dependencies. 

Pharo is supported by a strong community that grows daily. Pharo's community is active, innovative, and is always pushing limits of software engineering. The Pharo community consists of companies producing software, casual programmers but also high-level consultants, researchers, and teachers.
This book exists because of the Pharo community and we naturally dedicate this book to this group of people that many of us consider as our second family.


%=================================================================
\section*{Acknowledgments}

We would like to thank various people who have contributed to this book. In particular, we would like to thank:
\begin{itemize}
\item Camillo Bruni for his participation in the Zero Configuration chapter.
\item Noury Bouraqadi and Luc Fabresse for the Socket chapter.
\item Alain Plantec for his effort in the Setting Framework chapter and its effort to integrate it into Pharo.
\item Oscar Nierstrasz for writing and co-editing some chapters such as Regex and Monticello.
\item Dale Henrichs and Mariano Martinez Peck for their participation in the Metacello chapter.
\item Tudor Doru Girba for the Glamour chapter and the first documentation.
\item Cl\'ement Bera for his effort on the Exception chapter.
\item Nicolas Cellier for his participation in the Fun with Floats chapter.
\item Lukas Renggli for PetitParser and his work on the refactoring engine and smallLint rules. 
\item Jan Kurs and Guillaume Larcheveque for their participation in the PetitParser chapter.
\item Colin Putney for the initial version of FileSystem and Camillo Bruni for his review of FileSystem and his rewrite of the Pharo Core.
\item Vanessa Pe\~na for her participation in the Roassal and Mondrian chapters.
\item Renato Cerro for his help in proofreading.
\item You, for your questions, support, bug fixes, contribution, and encouragement.
\end{itemize}

We would like to also thank Hernan Wilkinson and Carlos Ferro for their reviews, Nicolas Cellier for the feedback on the number chapter, and Vassili Bykov for permission to adapt his Regex documentation

We thank Inria Lille Nord Europe for supporting this open-source project and for hosting the web site of this book. We also thank Object Profile for sponsoring the cover.

And last but not least, we also thank the Pharo community for its enthusiastic support of this project, and for informing us of the errors found in the first edition of this book.

We are also grateful to our respective institutions and national research agencies for their support and offered facilities. In particular, we thanks Program U-INICIA 11/06 VID 2011, University of Chile, and FONDECYT project 1120094. We also thank the Plomo \'Equipe Associ\'ee.

%=============================================================
\ifx\wholebook\relax\else
   \bibliographystyle{jurabib}
   \nobibliography{scg}
   \end{document}
\fi
%=============================================================
