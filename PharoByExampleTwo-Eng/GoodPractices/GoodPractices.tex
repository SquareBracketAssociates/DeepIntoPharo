% $Author: jannik $
% $Date: 2010-05-13 23:11:04 +0200 (Thu, 13 May 2010) $
% $Revision: 32942 $

% HISTORY:
% 2011-12-24 - Stef started chapter

%=================================================================
\ifx\wholebook\relax\else
% --------------------------------------------
% Lulu:
	\documentclass[a4paper,10pt,twoside]{book}
	\usepackage[
		papersize={6.13in,9.21in},
		hmargin={.75in,.75in},
		vmargin={.75in,1in},
		ignoreheadfoot
	]{geometry}
	\input{../common.tex}
	\pagestyle{headings}
	\setboolean{lulu}{true}
% --------------------------------------------
% A4:
%	\documentclass[a4paper,11pt,twoside]{book}
%	\input{../common.tex}
%	\usepackage{a4wide}
% --------------------------------------------
    \graphicspath{{figures/} {../figures/}}
	\begin{document}
	% \renewcommand{\nnbb}[2]{} % Disable editorial comments
	\sloppy
\fi
%=================================================================

%=================================================================


\chapter{Some Good Coding Practices}
\chalabel{goodPractices}


In this chapter we present some simple and good practices that make your code often more efficient and 
avoid generating unnecessary garbage. 

\section*{String concatenation}

In Smalltalk, the message \ct{,} \mthindex{String}{,} concatenates two strings. It is handy but this message is really costly since it copies the receiver. Therefore avoid it as much as possible. 

Prefer to use \ct{streamContents:} \mthindex{String}{streamContents:}, \ct{nextPut:} and \ct{nextPutAll:} since they avoid the duplication. The method \ct{streamContents:} expects a block would argument is a stream on the receiver. 
Such argument 

\begin{code}{}
String streamContents: [:s |
	s nextPutAll: 'aa'; nextPutAll: 'bb'.
	s contents]
\end{code}


Preallocate when possible the string used with\ct{new:streamContents:}. For example, the following method does not take advantage that the result is always a 8 characters string. 

\begin{code}{}
Time>>print24
 	"Return as 8-digit string 'hh:mm:ss', with leading zeros if needed"
 	^String streamContents:
 		[ :aStream | self print24: true on: aStream ]
\end{code}

This version does the same but more efficiently, since the system does not have to reallocate the underlying buffer. 
\begin{code}{}
Time>>print24
 	"Return as 8-digit string 'hh:mm:ss', with leading zeros if needed"
 	^ String new: 8 streamContents: [ :aStream | 
		self print24: true on: aStream ]
\end{code}		

\paragraph{Avoid }

When you can use \ct{nextPut:} use it instead of \ct{nextPutAll:}. Indeed \ct{nextPutAll:} requires that the argument is a container of elements. When you have already the element, no need to create an extra container.

For example better use the second form.
\begin{code}{}
stream nextPutAll: '0'
\end{code}

\begin{code}{}
stream nextPut: $0
\end{code}


\section*{Avoid creating temporary objects}

The computation of minute in Time could be written as \ct{asDuration minutes}. However, this solution
creates a duration object that is only used to get the minutes. 
In addition to be slow such approach generates extra garbage which stresses the garbage collector.

\begin{code}{}
Time>>minute
	^ self asDuration minutes
	
Time>>asDuration
	"Answer the duration since midnight"
	^ Duration seconds: seconds nanoSeconds: nanos 	
\end{code}

A much better solution is to use the encapsulation of the class Time and performs the computation locally as

\begin{code}{}
Time>>minute
	"Answer a number that represents the number of complete minutes in the receiver,
	after the number of complete hours has been removed."
	^ (seconds rem: SecondsInHour) quo: SecondsInMinute 
\end{code}

\section{Avoid several iterations on the same collection}
It may seem obvious but it is better to iterate once than two on the same collection, when we can do what we want in a single pass. 

For example, the following code creates a first string then creates another one where the character $: is removed.  

\begin{code}{}
Time>>hhmm24
 	"Return a string of the form 1123 (for 11:23 am), 2154 (for 9:54 pm), of exactly 4 digits"
 
 	^(String streamContents: 
 		[ :aStream | self print24: true showSeconds: false on: aStream ])
 			copyWithout: $:
\end{code}

Better implement it as 


\begin{code}{}
Time>>hhmm24
 	"Return a string of the form 1123 (for 11:23 am), 2154 (for 9:54 pm), of exactly 4 digits"
 
 	^ String new: 4 streamContents: [ :aStream | 
		self hour printOn: aStream base: 10 length: 2 padded: true.
		self minute printOn: aStream base: 10 length: 2 padded: true ]
\end{code}




%=================================================================

\ifx\wholebook\relax\else\end{document}\fi
%=================================================================

