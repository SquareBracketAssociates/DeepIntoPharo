% $Author:  $
% $Date: 2009-07-06 10:21:47 +0200 (Mon, 06 Jul 2009) $
% $Revision: 27886 $
% HISTORY:

%=================================================================
\ifx\wholebook\relax\else
% --------------------------------------------
% Lulu:
	\documentclass[a4paper,10pt,twoside]{book}
	\usepackage[
		papersize={6.13in,9.21in},
		hmargin={.75in,.75in},
		vmargin={.75in,1in},
		ignoreheadfoot
	]{geometry}
	\input{../common.tex}
	\pagestyle{headings}
	\setboolean{lulu}{true}
% --------------------------------------------
% A4:
%	\documentclass[a4paper,11pt,twoside]{book}
%	\input{../common.tex}
%	\usepackage{a4wide}
% --------------------------------------------
    \graphicspath{{figures/} {../figures/}}
	\begin{document}
	% \renewcommand{\nnbb}[2]{} % Disable editorial comments
	\sloppy
	\frontmatter
\fi
%=================================================================
\chapter{Biographies}\chalabel{Biographies}

\begin{wrapfigure}[10]{r}{1.1in}
\centering
\includegraphics[width=1in]{alexandre}
\end{wrapfigure}

Alexandre Bergel\footnote{\url{http://bergel.eu}} is Assistant Professor at the Department of Computer Science, Pleiad Laboratory, at the University of Chile, in Santiago.
Alexandre obtained his PhD in 2005 from the University of Berne, Switzerland. His PhD has been awarded by the prestigious Ernst-Denert prize in 2006. After his PhD, he completed a first postdoc at Lero \& Trinity College Dublin, Ireland, and a second at the Hasso-Plattner Institute, Germany. Alexandre and his collaborators carry out research in software engineering and software quality, more specifically on code profiling, testing and data visualization. Alexandre has authored over 60 articles, published in international and peer reviewed scientific forums, including the most competitive conferences and journals in the field of software engineering. Alexandre has participated to over 50 program committees of international events. Alexandre has also a strong interest in applying his research results to industry. Several of his research prototypes have been turned into products.

%%%%%%%
%Damien

\begin{wrapfigure}[10]{r}{1.1in}
\centering
\includegraphics[width=1in]{damien}
\end{wrapfigure}
%%%%%%%
Damien Cassou\footnote{\url{http://damiencassou.seasidehosting.st}} is an associate professor (\textit{ma\^itre de
  conf\'erences}) at the University of Lille~1, France, and a member
of the RMoD research group (Inria, LIFL). The main goal of his
research is to solve problems faced by developers everyday, from
browsing complex source code to semi-automatically decomposing large
commits. Before joining RMoD, Damien Cassou got his Ph.D. in computer
science from the University of Bordeaux I: his thesis was about
bringing general-purpose programming tools to dedicated domains
through a domain-specific architecture description language and a
programming framework generator. He is one of the developers of Pharo
and he collaborated on the Pharo by Example book.


\begin{wrapfigure}[10]{r}{1.1in}
\centering
\includegraphics[width=1in]{stephane}
\end{wrapfigure}
%\newpage
%%%%%%%
St\'ephane Ducasse\footnote{\url{http://stephane.ducasse.free.fr}} is directeur de recherche at Inria. Since 2011, he is scientific deputee of the Inria Lille Nord Europe 
research center where he leads the RMoD (\url{http://rmod.lille.inria.fr}) team. He is expert in two domains: object-oriented language design and reengineering.  He worked on traits, composable groups of methods, and this work got some impact. Traits have been introduced in AmbiantTalk, Racket, Squeak/Pharo, Perl, PHP and  under a variant into Scala, Fortress of SUN Microsystems. He is one of the developer of Pharo. He is also expert on software quality, program understanding, program visualizations, reengineering and metamodeling. He is one of the developer of Moose, an open-source software analysis platform (\url{http://www.moosetechnology.org}).  St\'ethane works with Synectique (\url{http://www.synectique.eu}) a company building dedicated tools for advanced software analysis.


%%%%%%%
\begin{wrapfigure}[10]{r}{1.1in}
\centering
\includegraphics[width=1in]{jannik}
\end{wrapfigure}
%Jannik 
Jannik Laval\footnote{\url{http://www.jannik-laval.eu}} is an associate professor at Mines-Telecom Institute, Mines Douai, France, since 2012. He received the doctorate degree in computer science from the University Lille 1, France, in June 2011. His thesis is about software quality, visualizations, and reengineering. He uses Moose for all his software analysis (\url{http://www.moosetechnology.org}).
%After his PhD, he completed a post-doc at LaBRI, Bordeaux, France.
At Mines Douai, he works on software engineering for embedded systems, and more particularly on modularity and tools for multi-robot systems. He is the main developer of Phratch, a visual programming language on top of Pharo (\url{http://car.mines-douai.fr/category/phratch/}). He uses Phratch for teaching robotics software engineering to engineer students. 



%=============================================================
\ifx\wholebook\relax\else
   \bibliographystyle{jurabib}
   \nobibliography{scg}
   \end{document}
\fi
%=============================================================
