% $Author: Benjamin $
% $Date: 2011-12-21 14:28:33 +0200 $
% $Revision: 28563 $

%=================================================================
\ifx\wholebook\relax\else
% --------------------------------------------
% Lulu:
	\documentclass[a4paper,10pt,twoside]{book}
	\usepackage[
		papersize={6.13in,9.21in},
		hmargin={.75in,.75in},
		vmargin={.75in,1in},
		ignoreheadfoot
	]{geometry}
	\input{../common.tex}
	\setboolean{lulu}{true}
    \newcommand{\cb}[1]{\nnbb{Camillo}{#1}} % Camillo Bruni
% --------------------------------------------
% A4:
%	\documentclass[a4paper,11pt,twoside]{book}
%	\input{../common.tex}
%	\usepackage{a4wide}
% --------------------------------------------
    \graphicspath{{figures/} {./figures/}}
	\begin{document}
\fi
%=================================================================
%\renewmessage{\nnbb}[2]{} % Disable editorial comments
\sloppy
%=================================================================

\chapter{Zero Configuration}


\section{Getting the VM and the Image}

\begin{code}{}
wget http://pharo.gforge.inria.fr/ci/script/ciPharo20PharoVM.sh
\end{code}


\begin{code}{}
bash ./ciPharo20PharoVM.sh --help 
\end{code}

\begin{code}{}
This script will download the latest Pharo 2.0 image and VM

Result in the current directory:
    vm               VM directory
    vm.sh            Script forwarding to the VM in vm
    Pharo.image      The latest pharo image
    Pharo.changes    The corresponding pharo changes
\end{code}

%% in the vm directory

\paragraph{Grabbing it and executing it}

\begin{code}{}
curl http://pharo.gforge.inria.fr/ci/script/ciPharo20PharoVM.sh | bash
\end{code}


\section{Getting the latest VM only}

\begin{code}{}
wget http://pharo.gforge.inria.fr/ci/script/ciPharoVM.sh
\end{code}

\begin{code}{}
bash ciPharoVM.sh --help


This script will download the latest Pharo VM

Result in the current directory:
    vm               directory containing the VM
    vm.sh            script forwarding to the VM inside vm
\end{code}





\section{The scripts}

\begin{figure}[!h]
	\centering
	\includegraphics[width=\textwidth]{webSite}
	\caption{All the scripts are available at \ct{http://pharo.gforge.inria.fr/ci/script/} \label{fig:website}.}
\end{figure}
	



\paragraph{ciPharo20PharoVM.sh.}

\begin{scriptsize}
\begin{verbatim}
#!/bin/bash

# stop the script if a single command fails
set -e 

# ARHUMENT HANDLING ===========================================================

if { [ "$1" = "-h" ] || [ "$1" = "--help" ]; }; then
    echo "This script will download the latest Pharo 2.0 image and VM

Result in the current directory:
    vm               VM directory
    vm.sh            Script forwarding to the VM in vm
    Pharo.image      The latest pharo image
    Pharo.changes    The corresponding pharo changes"
    exit 0
elif [ $# -gt 0 ]; then
    echo "--help is the only argument allowed"
    exit 1
fi

# FETCH DATA ==================================================================
wget --quiet -qO - http://pharo.gforge.inria.fr/ci/script/ciPharoVM.sh | bash
wget --quiet -qO - http://pharo.gforge.inria.fr/ci/script/ciPharo20.sh | bash
\end{verbatim}
\end{scriptsize}

\section{Conclusion}
As conclusion, you have now tools to quickly create UIs and to reuse them. Since the reuse is really a strong value for specs, keep in mind that your tools can be reused. So do not forget to provide a proper API for it.

%=========================================================
\ifx\wholebook\relax\else
   \bibliographystyle{jurabib}
   \nobibliography{scg}
   \end{document}
\fi

%=================================================================
\ifx\wholebook\relax\else\end{document}\fi
%=================================================================

%-----------------------------------------------------------------

%%% Local Variables:
%%% coding: utf-8
%%% mode: latex
%%% TeX-master: t
%%% TeX-PDF-mode: t
%%% ispell-local-dictionary: "english"
%%% End: